%! TEX program = xelatex
% {{{ PREAMBLE
\documentclass[10pt,aspectratio=169]{beamer}

\graphicspath{{img/}{../img/}}
\usepackage[T1]{fontenc}
\usepackage[english]{babel}

\usepackage{microtype}
\usepackage{booktabs}
\usepackage{multirow}

\usepackage{enumerate}
\usepackage{listings}
\usepackage{xcolor}
\usepackage{tikz}

% {{{ colors
\setbeamercolor{background canvas}{bg=white}
\newcommand{\mlbrown}[1]{\textcolor{mLightBrown}{#1}}
\newcommand{\mdbrown}[1]{\textcolor{mDarkBrown}{#1}}
\newcommand{\mlgreen}[1]{\textcolor{mLightGreen}{#1}}
\newcommand{\mdteal}[1]{\textcolor{mDarkTeal}{#1}}
% Available aliases: \<color>{<text>}
%
%   tlblue tlred tlpurple tlgreen tlbrown
%   tdblue tdred tdpurple tdgreen tdbrown
%
\usepackage{pgfplots}
\usepackage{pgfplotsthemetol}
\newcommand{\tlblue}[1]{\textcolor{TolLightBlue}{#1}}
\newcommand{\tdblue}[1]{\textcolor{TolDarkBlue}{#1}}
\newcommand{\tlred}[1]{\textcolor{TolLightRed}{#1}}
\newcommand{\tdred}[1]{\textcolor{TolDarkRed}{#1}}
\newcommand{\tlpurple}[1]{\textcolor{TolLightPurple}{#1}}
\newcommand{\tdpurple}[1]{\textcolor{TolDarkPurple}{#1}}
\newcommand{\tlpink}[1]{\textcolor{TolLightPink}{#1}}
\newcommand{\tdpink}[1]{\textcolor{TolDarkPink}{#1}}
\newcommand{\tlgreen}[1]{\textcolor{TolLightGreen}{#1}}
\newcommand{\tdgreen}[1]{\textcolor{TolDarkGreen}{#1}}
\newcommand{\tlbrown}[1]{\textcolor{TolLightBrown}{#1}}
\newcommand{\tdbrown}[1]{\textcolor{TolDarkBrown}{#1}}
% }}}

% math
\usepackage{amsmath}
\renewcommand{\epsilon}{\varepsilon}
\renewcommand{\theta}{\vartheta}
\renewcommand{\rho}{\varrho}
\renewcommand{\phi}{\varphi}
\usepackage{amsfonts}
\usepackage{bbold}
\usepackage[mathrm=sym]{unicode-math}
\setmathfont{Fira Math}

%!TEX root = ../main.tex

\newcommand{\n}           {$\upnu$}
\newcommand{\x}           {$\upchi$}
\renewcommand{\b}         {$\upbeta$}
\newcommand{\g}           {$\upgamma$}
\renewcommand{\a}         {$\upalpha$}
\newcommand{\NAZ}         {$\mathcal{N}(A,Z)$}
\newcommand{\cpt}         {$CPT$}
\newcommand{\aofM}        {{\mathring{a}_\text{of}^{(3)}}}
\newcommand{\aof}         {$\aofM$}
\newcommand{\aofooM}      {{(a_\text{of}^{(3)})_{00}}}
\newcommand{\aofoo}       {$\aofooM$}
\newcommand{\ga}          {$g_\alpha$}

\newcommand{\fillme}   [1]{\textcolor{red}{\texttt{#1}}}
\newcommand{\reg}         {\textsuperscript{\textregistered}}
\newcommand{\chapendgliph}{%
  \vspace{22pt}
  \begin{center}
    \fontsize{26}{26}\darkred\char"2766
  \end{center}
}

% data sets
\newcommand{\m}        [1]{\texttt{#1}}
\newcommand{\bege}        {\textsc{BEGe}}
\newcommand{\scoax}       {\textsc{SemiCoax}}
\newcommand{\icoax}       {\textsc{InvCoax}}
\newcommand{\GD}       [1]{\m{GD#1}}
\newcommand{\ANG}      [1]{\m{ANG#1}}
\newcommand{\GTF}      [1]{\m{GTF#1}}
\newcommand{\RG}       [1]{\m{RG#1}}
\newcommand{\IC}       [1]{\m{IC#1}}
\newcommand{\pplus}       {p$^+$}
\newcommand{\nplus}       {n$^+$}
\newcommand{\run}[1]      {\textsc{run~\oldstylenums{#1}}}

% global model
\newcommand{\Mone}        {\m{M1}}
\newcommand{\Mtwo}        {\m{M2}}
\newcommand{\enrBEGeII}   {\m{M1-BEGe}}
\newcommand{\enrCoaxII}   {\m{M1-SemiCoax}}
\newcommand{\enrGeII}     {\m{M2-AllEnr}}
\newcommand{\enrBEGeIIp}  {\m{M1-BEGe\textsuperscript{+}}}
\newcommand{\enrSCoaxIIp} {\m{M1-SemiCoax\textsuperscript{+}}}
\newcommand{\enrICoaxIIp} {\m{M1-InvCoax\textsuperscript{+}}}
\newcommand{\enrGeIIp}    {\m{M2-AllEnr\textsuperscript{+}}}
% K model
\newcommand{\Mokvn}{\m{M1-K40}}
\newcommand{\Mtkvn}{\m{M2-K40}}
\newcommand{\Mokvz}{\m{M1-K42}}
\newcommand{\Mtkvz}{\m{M2-K42}}
\newcommand{\MoSBon}{\m{M1-SB1}}
\newcommand{\MtSBon}{\m{M2-SB1}}
\newcommand{\MoSBtw}{\m{M1-SB2}}
\newcommand{\MtSBtw}{\m{M2-SB2}}
\newcommand{\MoSBth}{\m{M1-SB3}}
\newcommand{\MtSBth}{\m{M2-SB3}}

\newcommand{\annmse}         {ANN\textinf{MSE}}
\newcommand{\deltae}         {$\delta{E}$}
\newcommand{\aoe}            {$A/E$}
\newcommand{\measurementM}[3]{{#1}\substack{+#2 \\ -#3}}
\newcommand{\measurement} [3]{$\measurementM{#1}{#2}{#3}$}
\newcommand{\stat}        [1]{{#1}_\text{stat}}
\newcommand{\syst}        [1]{{#1}_\text{sys}}
\newcommand{\pdf}            {{PDF}}
\newcommand{\pvalue}         {$p$-value}

% units
\newcommand{\ctsper}      {$\text{cts}/(\text{keV}{\cdot}\text{kg}{\cdot}\text{yr})$}
\newcommand{\powctsper}[1]{{$10^{#1}$~\ctsper}}
\newcommand{\pIbi}        {{$1.2\cdot10^{-2}$~\ctsper}}
\newcommand{\pIIbi}       {{$10^{-3}$~\ctsper}}
\newcommand{\kgyr}        {{kg$\cdot$yr}}
\newcommand{\mubq}        {{$\upmu$Bq}}
\newcommand{\mum}         {{$\upmu$m}}
\newcommand{\mus}         {{$\upmu$s}}
%\newcommand{\mubqperkg}   {{$\frac{\upmu\mathrm{Bq}}{\mathrm{kg}}$}}
\newcommand{\mubqperkg}   {{${\upmu\mathrm{Bq}}/{\mathrm{kg}}$}}
\newcommand{\powtenyr} [1]{$10^{#1}$~yr}

% dbd
\newcommand*{\ub}{\upbeta}
\newcommand*{\un}{\upnu}
\newcommand*{\ux}{\upchi}
\newcommand{\qbbM}        {Q_{\ub\ub}}
\newcommand{\qbb}         {$\qbbM$}
\newcommand{\thalfzeroM}  {{T^{0\un}_{1/2}}}
\newcommand{\thalftwoM}   {{T^{2\un}_{1/2}}}
\newcommand{\thalfzero}   {$\thalfzeroM$}
\newcommand{\thalftwo}    {$\thalftwoM$}
\newcommand{\thalflvM}    {{T^{2\un_\text{LV}}_{1/2}}}
\newcommand{\thalflv}     {$\thalflvM$}
\newcommand{\thalfmajo}   {{${T^{0\un\ux(\ux)}_{1/2}}$}}
\newcommand{\thalfzx}     {{${T^{0\un\ux}_{1/2}}$}}
\newcommand{\thalfzxx}    {{${T^{0\un\ux\ux}_{1/2}}$}}
\newcommand{\nmez}        {$\mathcal{M}^{0\un}$}
\newcommand{\nmet}        {$\mathcal{M}^{2\un}$}
\newcommand{\nmemajo}     {$\mathcal{M}^{0\un\ux(\ux)}$}
\newcommand{\psfz}        {$G^{0\un}$}
\newcommand{\psft}        {$G^{2\un}$}
\newcommand{\psfmajo}     {$G^{0\un\ux(\ux)}$}
\newcommand{\mbb}         {$\langle{m_{\ub\ub}}\rangle$}
\newcommand{\mb}          {$m_{\ub}$}
\newcommand{\nbbM}        {{\un\ub\ub}}
\newcommand{\onbbM}       {{0\nbbM}}
\newcommand{\onbb}        {$\onbbM$}
\newcommand{\onbbp}       {$0\un\ub^+\ub^+$}
\newcommand{\onbbm}       {$0\un\ub^-\ub^-$}
\newcommand{\nnbbM}       {{2\nbbM}}
\newcommand{\nnbb}        {$2\nbbM$}
\newcommand{\nnbbp}       {$2\un\ub^+\ub^+$}
\newcommand{\nnbbm}       {$2\un\ub^-\ub^-$}
\newcommand{\onbbxM}      {{0\nbbM\ux}}
\newcommand{\onbbx}       {$\onbbxM$}
\newcommand{\onbbxx}      {$0\nbbM\ux\ux$}
\newcommand{\nnbblvM}     {{2\nbbM\textinf{LV}}}
\newcommand{\nnbblv}      {$\nnbblvM$}
\newcommand{\onecec}      {$0\un\text{ECEC}$}
\newcommand{\nnecec}      {$2\un\text{ECEC}$}

\newcommand{\thalfzerolimit}{1.8 \cdot 10^{26}}
\newcommand{\thalftwovalue}{(2.050 \pm \stat{0.013} \pm \syst{0.044}) \cdot 10^{21}}
\newcommand{\thalftwovalueshort}{(2.050 \pm 0.046) \cdot 10^{21}}
\newcommand{\thalfmajobestlimit}{7.5 \cdot 10^{23}}

\newcommand{\gexpophasetwo}    {60.1~\kgyr}
\newcommand{\gexpophasetwobkg} {61.4~\kgyr}
\newcommand{\gexpophasetwop}   {43.6~\kgyr}
\newcommand{\gexpophasetwopbkg}{44.1~\kgyr}
\newcommand{\gexpobkg}         {105.5~\kgyr}
\newcommand{\gexpo}            {103.7~\kgyr}

\newcommand{\gebkgBEGeII}      {32.7~\kgyr}
\newcommand{\gebkgSCoaxII}     {28.6~\kgyr}
\newcommand{\gebkgBEGeIIp}     {22.2~\kgyr}
\newcommand{\gebkgSCoaxIIp}    {13.2~\kgyr}
\newcommand{\gebkgICoaxIIp}    {8.8~\kgyr}

% experiments and equipment
\newcommand{\gerda}       {\textsc{Gerda}}
\newcommand{\phaseone}    {Phase~I}
\newcommand{\phasetwo}    {Phase~II}
\newcommand{\phasetwop}   {Phase~II$^+$}
\newcommand{\gerdatwo}    {\gerda\ Phase~II}
\newcommand{\LArGe}       {\textsc{LArGe}}
\newcommand{\majorana}    {\textsc{Majorana}}
\newcommand{\majoranademo}{\textsc{Majorana Demonstrator}}
\newcommand{\igex}        {\textsc{Igex}}
\newcommand{\hdm}         {\textsc{HdM}}
\newcommand{\kamlandzen}  {\textsc{KamLAND-Zen}}
\newcommand{\cuoricino}   {\textsc{Cuoricino}}
% code
\newcommand{\geant}       {\textsc{Geant4}}
\newcommand{\GEANT}       {\textsc{\mbox{{Geant}}}}
\newcommand{\rootv}       {\textsc{Root}}
\newcommand{\CERN}        {{\mbox{\textsc{Cern}}}}
\newcommand{\mage}        {\textsc{MaGe}}
\newcommand{\gelatio}     {\textsc{Gelatio}}
\newcommand{\mgdo}        {\mbox{MGDO}}
\newcommand{\tier}        {\textsc{Tier}}
\newcommand{\decayzero}   {\textsc{Decay0}}
% isotopes
\newcommand{\nuc}      [2]{\textsu{#2}{#1}}
\newcommand{\gesix}       {\nuc{Ge}{76}}
\newcommand{\kvn}         {\nuc{K}{40}}
\newcommand{\kvz}         {\nuc{K}{42}}
\newcommand{\Am}          {\nuc{Am}{241}}
\let\Rn\undefined\newcommand{\Rn}{\nuc{Rn}{222}}
\newcommand{\Ra}          {\nuc{Ra}{226}}
\newcommand{\Po}          {\nuc{Po}{210}}
\newcommand{\Arh}         {\nuc{Ar}{42}}
\newcommand{\Arl}         {\nuc{Ar}{39}}
\newcommand{\Kr}          {\nuc{Kr}{85}}
\newcommand{\Ba}          {\nuc{Ba}{133}}
\newcommand{\Bil}         {\nuc{Bi}{212}}
\newcommand{\Bih}         {\nuc{Bi}{214}}
\newcommand{\Thh}         {\nuc{Th}{232}}
\newcommand{\Th}          {\nuc{Th}{228}}
\newcommand{\Tl}          {\nuc{Tl}{208}}
\newcommand{\Ul}          {\nuc{U}{235}}
\newcommand{\Uh}          {\nuc{U}{238}}
\newcommand{\Co}          {\nuc{Co}{60}}
\newcommand{\Ac}          {\nuc{Ac}{228}}
\newcommand{\Pbl}         {\nuc{Pb}{210}}
\newcommand{\Pbh}         {\nuc{Pb}{214}}
\newcommand{\Pa}          {\nuc{Pa}{234\text{m}}}
\newcommand{\Zn}          {\nuc{Zn}{65}}

\newcommand{\expoM}      {\mathcal{E}}
\newcommand{\expo}       {$\expoM$}
\newcommand{\expoGesixAV}{$\expoM_{76}^\text{AV}$}


\newcommand{\mytitle}{Search for new physics with two-neutrino double-beta decay in \gerda}
\newcommand{\people}{L.~Pertoldi}
\institute{Universit\`a degli Studi di Padova / INFN Padova}
\newcommand{\mydate}{February 2021}
\newcommand{\place}{Ph.D.~defense, \textit{On a remote server}}

% links
\definecolor{TolDARKBlue}{HTML}{3973AC}
\hypersetup{%
  colorlinks = true,
  urlcolor = TolDARKBlue,
  linkcolor = mDarkTeal
}
\renewcommand\UrlFont{\color{TolDARKBlue}\ttfamily\small}
\newcommand{\doi}[1]{\href{https://doi.org/#1}{\footnotesize\ttfamily #1}}
\newcommand{\arxiv}[1]{\href{https://arxiv.org/abs/#1}{\footnotesize\ttfamily arXiv:#1}}

% metropolis theme options
\usetheme[numbering=fraction,block=fill]{metropolis}
\setbeamertemplate{frame footer}{\emph{\mytitle} $\bullet$ \people\ $\bullet$ \mydate}
\title{\mytitle}
% \subtitle{Ph.D.~Defense}
\titlegraphic{%
  \vspace{4.6cm}%
    \begin{minipage}[c]{\textwidth}
      \hfill%
      \raisebox{-0.5\height}{\includegraphics[height=1.7cm]{img/logos/unipd-logo.pdf}}%
      \hspace{0.3cm}%
      \raisebox{-0.5\height}{\includegraphics[height=1.7cm]{img/logos/infn-logo.pdf}}%
      \hspace{0.3cm}%
      \raisebox{-0.5\height}{\includegraphics[height=1.7cm]{img/logos/gerda-logo.png}}%
    \end{minipage}
}
\date{\place\ \sep\ \mydate}
\author{\people}

% {{{ customization
\usepackage{appendixnumberbeamer}
\setbeamercovered{dynamic}
% for heavier glyphs
\setsansfont[BoldFont={Fira Sans SemiBold}]{Fira Sans Book}
\setmonofont{Fira Code}[Contextuals=Alternate, Scale=MatchLowercase]
% usage: \begin{simpleblock} ... \end{simpleblock}
\usepackage[most]{tcolorbox}
\newtcolorbox{simpleblock}{
%  enhanced,  % does not work with tikz/external
  boxsep=0.25ex,
%  arc=1.25ex,
  arc=0ex,
  opacityframe=.6,
  opacityback=.6,
  colback=bg!80!fg,
  coltext=mDarkTeal,
  colframe=white,
  boxrule=0pt,
}
% }}}

\newcommand{\arrow}{$\longmapsto$}
\newcommand{\sep}{$\bullet$}
% }}}
\begin{document}
\maketitle
\begin{frame}{Outline}
  \begin{enumerate}
    \item \gerda: science goal and experimental design
    \item The background model before analysis cuts
    \item The background model after the LAr veto cut
    \item Precision analysis of the \nnbb\ event distribution
    \item Search for new physics
  \end{enumerate}
  \begin{center}
    \includegraphics[width=0.7\textwidth]{plots/bkg/raw/combined-results-M1only.pdf}
  \end{center}
\end{frame}
\begin{frame}{\b\ and \b\b\ decays}
  \begin{center}
    \begin{tikzpicture}[font=\small]
      \node at (0,0) {\includegraphics[width=0.6\textwidth]{bb-artist.png}};

      \node[align=center] at (-3.5,-1.9) {\scshape $\upbeta$ decay};
      \node[align=center] at ( 0.0,-1.9) {\scshape double-$\upbeta$ decay};
      \node[align=center] at ( 3.5,-1.9) {\scshape neutrinoless \\ \scshape double-$\upbeta$ decay};

      \node at (-4.3, 1.5) {$e^-$};
      \node at (-2.2,-1.1) {\footnotesize $\overline{\upnu}_e$};

      \node at (-1.2, 1.1) {$e^-$};
      \node at ( 1.4, 1.2) {$e^-$};
      \node at ( 1.3,-1.0) {\footnotesize $\overline{\upnu}_e$};
      \node at (-1.5,-0.9) {\footnotesize $\overline{\upnu}_e$};

      \node at ( 3.2, 1.5) {$e^-$};
      \node at ( 4.6,-1.2) {$e^-$};
    \end{tikzpicture}
  \end{center}
\end{frame}
\section{The \gerda\ experiment}
\begin{frame}{Why studying the double-$\beta$ decay?}
  The search for \onbb\ \alert{is not a mere measurement of neutrino's intrinsic
  properties}, just look at the huge literature production\footnote{W.~Rodejohann,
  \m{IJMP, E 20 (2011) 1833}}
  \begin{itemize}
    \item Lepton number $\longleftrightarrow$ Barion number \arrow\
      \alert{GUTs}, \alert{baryogenesis} (not guaranteed!)
    \item (almost always) Majorana mass term predicted (\emph{black-box}
      theorem)
    \item provides access to lots of fundamental parameters, shared or not with
      other experimental techniques
    \item \alert{standard interpretation}: \emph{the neutrino that mediates
      \onbb\ is the one that oscillates and the Standard Model is an effective
      theory of some GUT} (seesaw mechanism)
    \begin{itemize}
      \item Connection with the Majorana effective mass:
        $(T^{0\nu}_{1/2})^{-1}=G_{0\nu} |\mathcal{M}_{0\nu}|^2 m_{\beta\beta}^2$
        \arrow\ Oscillation parameters and absolute neutrino mass scale
    \end{itemize}
  \item countless non-standard interpretations$^1$
  \end{itemize}
\end{frame}
\begin{frame}[plain]{\alert{GER}manium \alert{D}etector \alert{A}rray}
  \begin{columns}[c]
    \column{0.3\textwidth}%
      \begin{simpleblock}
        \gerda\ looks for \onbb\ with enriched, high-purity \gesix\
        detectors \\ \alert{source $=$ detector}
      \end{simpleblock}
      \begin{itemize}
        \item Installed at LNGS (3500 m.w.e.), in activity since 2009
          \arrow\ \alert{\phaseone}
        \item Hardware upgrade 2015 \arrow\ \alert{\phasetwo}
        \item Decommissioned in 2020 \arrow\ \alert{LEGEND-200}
      \end{itemize}
    \column{0.7\textwidth}%
      \begin{center}
        \begin{tikzpicture}[fill=bg]
          \node at (0,0) {\includegraphics[height=7cm]{setup/gerda-artist.jpg}};
          \node(e) at (3.5,3.5) [rectangle,draw,fill] {\textsc{scintillators}};
          \draw[thick,red,->] (e.south) .. controls +(0,-1) and +(0,1) .. (2.5,2.4);
          \node(a) at (1.6,-1.2) [rectangle,draw,fill] {\color{red}LAr};
          \node(b) at (3.2,1.6) [rectangle,draw,fill] {\gesix\ \textsc{detectors}};
          \draw[thick,red,->] (b.south) .. controls +(0,0) and +(1,0) .. (1.9,-0.4);
          \node(c) at (-1.5,1.6) [rectangle,draw,fill] {H$_2$O};
          \draw[thick,red,->] (c.south) .. controls +(0,0) and +(-1,0) .. (0,-1);
          \node(d) at (0,3) [rectangle,draw,fill] {Cu \textsc{shielding}};
          \draw[thick,red,->] (d.south) .. controls +(0,0) and +(-1,0) .. (1,0);
          \node at (3,-3) [rectangle,draw=mLightBrown, fill=white] {Hosted at LNGS};
        \end{tikzpicture}
      \end{center}
  \end{columns}
\end{frame}
\begin{frame}{The scintillation of liquid argon (LAr)}
  Liquid argon as a \emph{cooling medium}, \emph{passive shield} against
  backgrounds (\phaseone) and \alert{active shield} (\phasetwo).
  \begin{center}
    \includegraphics[width=0.8\textwidth]{lar-scint-mechanism.pdf}
  \end{center}
  \arrow\ light is produced at the passage of ionizing radiation,
  \emph{detector medium}
\end{frame}
\begin{frame}{Signal and background discrimination techniques}
  \begin{center}
    \includegraphics[width=\textwidth]{gedet/gerda-events.png}
  \end{center}
  \arrow\ no scintillation light is expected in signal events \\
  \arrow\ \alert{LAr veto cut:} Ge events with coincident LAr light are cut
\end{frame}
\begin{frame}{Collecting the LAr scintillation light}
  \tikzstyle{every picture}+=[remember picture]
  \tikzstyle{na} = [baseline=-.5ex]
  \begin{columns}%
    \column{0.55\textwidth}%
      \begin{itemize}
        \item<1-> 16 \alert{PMTs} \tikz[na] \node[coordinate] (n1) {};
        \item<2-> light-guiding \alert{fibers} + SiPM readout \tikz[na]
          \node[coordinate] (n2) {};
        \item<3-> nylon \alert{mini-shrouds} for each string \tikz[na]
          \node[coordinate] (n3) {}; \\
          \arrow\ mechanical barrier against \kvz\ ions
      \end{itemize}
    \column{0.45\textwidth}%
        \begin{tikzpicture}
          \node (t1) at (0,2.8)  {\includegraphics[width=2cm]{setup/larveto-top-pic.jpg}};
          \node (t2) at (0.0,0)  {\includegraphics[width=2cm]{setup/fibers-pic.pdf}};
          \node (t3) at (0,-2.6) {\includegraphics[width=2cm]{setup/larveto-bottom-pic.jpg}};
          \node (t4) at (2.5,0)  {\includegraphics[height=0.95\textheight]{setup/lar-veto-tikz.pdf}};
          \node (t5) at (-2.1,0) {\includegraphics[width=1.3cm]{setup/ms-pic.jpg}};
        \end{tikzpicture}
  \end{columns}
  \begin{tikzpicture}[overlay,very thick,color=mLightBrown]
    \path[->]<1> (n1) edge [out=0, in=160] (t1);
    \path[->]<1> (n1) edge [out=0, in=200] (t3);
    \path[->]<2> (n2) edge [out=0] (t2);
    \path[->]<3> (n3) edge [out=0] (t5);
  \end{tikzpicture}
\end{frame}
\section{The background model before analysis cuts}
\section{The background model after the LAr veto cut}
\section{Precision fit of \texorpdfstring{\nnbb}{2nubb} events: old and new physics}
\section{Closing out}
\begin{frame}{Future}
\end{frame}
\appendix
\begin{frame}[standout]{}
  Backup
\end{frame}
\end{document}
