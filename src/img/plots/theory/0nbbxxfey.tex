%! TEX program = lualatex
\documentclass{standalone}
\usepackage[lining]{ebgaramond}%[StylisticSet={7,9}]
\usepackage[math-style=ISO, bold-style=ISO]{unicode-math}
\setmathfont{Garamond-Math.otf}
\usepackage[compat=1.1.0]{tikz-feynman}
\newcommand\dy{0.5cm}
\newcommand\ldx{7cm}
\newcommand\lsx{2.5cm}
\newcommand\hei{1.1cm}
\newcommand\etoe{2.0cm}
\newcommand\ange{0.6cm}
\newcommand\lene{4cm}
\begin{document}
\LARGE
\tikzfeynmanset{%
   every edge/.style={line width=1.5pt},
}
\begin{tikzpicture}
  \begin{feynman}
    \vertex (b);
    \vertex[below=\hei+\etoe of b] (c);
    \vertex[below=0.33*\hei+0.33*\etoe of b] (x);
    \vertex[below=0.66*\hei+0.66*\etoe of b] (xx);
    \vertex[below left=\hei and \hei of c] (d);
    \vertex[above left=\hei and \hei of b] (a);
    % spectator quarks vertices
    \vertex[above=\dy of a] (e);
    \vertex[above=\dy of e] (f);
    \vertex[below=\dy of d] (g);
    \vertex[below=\dy of g] (h);
    % external legs
    \vertex[left=\lsx of a] (i1) {$d$};
    \vertex[left=\lsx of d] (i2) {$d$};
    \vertex[right=\ldx of a] (f1) {$u$};
    \vertex[right=\lene of b] (f2) {\color{red}$e^-$};
    \vertex[right=\lene of c] (f3) {\color{red}$e^-$};
    \vertex[right=\lene of x] (x1) {$\upchi$};
    \vertex[right=\lene of xx] (xx1) {$\upchi$};
    \vertex[right=\ldx of d] (f4) {$u$};
    % spectator quarks
    \vertex[right=\ldx of e] (f5) {$d$};
    \vertex[left=\lsx of e] (f6) {$d$};
    \vertex[right=\ldx of f] (f7) {$u$};
    \vertex[left=\lsx of f] (f8) {$u$};
    \vertex[right=\ldx of g] (f9) {$d$};
    \vertex[left=\lsx of g] (f10) {$d$};
    \vertex[right=\ldx of h] (f11) {$u$};
    \vertex[left=\lsx of h] (f12) {$u$};

    \diagram* {%
      (a) -- [boson, blue, edge label'=$W^-$] (b),
      (c) -- [boson, blue, edge label'=$W^-$] (d),
      (b) -- [green!60!black] (x),
      (x) -- [green!60!black, edge label'=$\upnu$] (xx),
      (xx) -- [green!60!black] (c),
      (x) -- [scalar] (x1),
      (xx) -- [scalar] (xx1),
      (i1) -- [fermion] (a),
      (i2) -- [fermion] (d),
      (a) -- [fermion] (f1),
      (b) -- [fermion, red] (f2),
      (c) -- [fermion, red] (f3),
      (d) -- [fermion] (f4),
      % spectator quarks
      (f6)  -- (e),
      (e)   -- (f5),
      (f8)  -- (f),
      (f)   -- (f7),
      (f10) -- (g),
      (g)   -- (f9),
      (f12) -- (h),
      (h)   -- (f11),
  };
  \end{feynman}
\end{tikzpicture}
\end{document}
