%!TEX root = ../main.tex

\chapter{Assay measurements}%
\label{apdx:assay}

To assess whether construction materials meet the \gerda\ radio-purity requirements,
screening measurement campaigns have been carried one before the \phasetwo\ and \phasetwop\
upgrades. The extracted radioactive decay activities constitute an important set of prior
information that guides the construction of the background model presented in
\cref{chap:bkg:raw:ph2} and following chapters. The results of the measurements are
converted to useful units for the characterisation of prior distributions in the
background model and other analyses. Part of these measurements are published also
in~\cite{Agostini2018a}.

The following groups of isotopes are assumed to be in secular equilibrium:
\[
  ^{238}\text{U}  \succ ^{234\text{m}}\text{Pa}               \;,\quad
  ^{226}\text{Ra} \succ ^{214}\text{Pb} \succ ^{214}\text{Bi} \;,\quad
  ^{228}\text{Ra} \succ ^{228}\text{Ac}                       \;,\quad
  ^{228}\text{Th} \succ ^{212}\text{Bi} \succ ^{208}\text{Tl} \;.
\]
For each measurement a recommendation is also given on how to properly use the values. 

\clearpage
\thispagestyle{empty}
\begin{sidewaystable}
  \caption{%
    Activity of \gerda\ \phasetwo\ components calculated from radio-purity screening
    measurements. Values in gray can be used as priors for \phasetwo\ background modeling
    as they are; values in yellow are (partly) derived from ICPMS/NAA measurements under
    the assumption of secular equilibrium in the U/Th decay chains; values in light blue
    have to be multiplied by 2/3. The latter is an approximation for the shortening of the
    cables before deployment in \gerda.
  }\label{tab:bkg:ph2:assay}
  \definecolor{TolGray}      {HTML}{BBBBBB}
\definecolor{TolPaleGray}  {HTML}{DDDDDD}

\definecolor{TolBriBlue}   {HTML}{4477AA}
\definecolor{TolBriCyan}   {HTML}{66CCEE}
\definecolor{TolBriGreen}  {HTML}{228833}
\definecolor{TolBriYellow} {HTML}{CCBB44}
\definecolor{TolBriRed}    {HTML}{EE6677}
\definecolor{TolBriPurple} {HTML}{AA3377}

\definecolor{TolHCYellow}  {HTML}{DDAA33}
\definecolor{TolHCRed}     {HTML}{BB5566}
\definecolor{TolHCBlue}    {HTML}{004488}

\definecolor{TolVibBlue}   {HTML}{0077BB}
\definecolor{TolVibCyan}   {HTML}{33BBEE}
\definecolor{TolVibTeal}   {HTML}{009988}
\definecolor{TolVibOrange} {HTML}{EE7733}
\definecolor{TolVibRed}    {HTML}{CC3311}
\definecolor{TolVibMagenta}{HTML}{EE3377}

\definecolor{TolMutIndigo} {HTML}{332288}
\definecolor{TolMutCyan}   {HTML}{88CCEE}
\definecolor{TolMutTeal}   {HTML}{44AA99}
\definecolor{TolMutGreen}  {HTML}{117733}
\definecolor{TolMutOlive}  {HTML}{999933}
\definecolor{TolMutSand}   {HTML}{DDCC77}
\definecolor{TolMutRose}   {HTML}{CC6677}
\definecolor{TolMutWine}   {HTML}{882255}
\definecolor{TolMutPurple} {HTML}{AA4499}

\definecolor{TolLigBlue}   {HTML}{77AADD}
\definecolor{TolLigCyan}   {HTML}{99DDFF}
\definecolor{TolLigMint}   {HTML}{44BB99}
\definecolor{TolLigPear}   {HTML}{BBCC33}
\definecolor{TolLigOlive}  {HTML}{AAAA00}
\definecolor{TolLigYellow} {HTML}{EEDD88}
\definecolor{TolLigOrange} {HTML}{EE8866}
\definecolor{TolLigPink}   {HTML}{FFAABB}

\newcommand{\rot}[1]{\begin{rotate}{50}#1\end{rotate}}
\newcommand{\rcsg}{\rowcolor{TolLigMint}}
\newcommand{\rclb}{\rowcolor{TolLigBlue}}
\newcommand{\ccsg}{\cellcolor{TolPaleGray}}
\newcommand{\ccw}{\cellcolor{white}}

\begin{tabular}{rllcccccc}
  \rowcolor{white}
  Part        & Material               & Method                    & {$^{228}$Ra (\mubq)}    & {\Ra\ (\mubq)}          & {\Th\ (\mubq)}           & {\Co\ (\mubq)}     & {\kvn\ (mBq)}     & {\Uh\ (mBq)}         \\
  \midrule
  \ccw holders& \ccw Si V IKZ          & \ccw $\gamma$ spec.       & $<250$                  & \ccw $<130$             & \ccw $<96$               & $<100$             & $2.75\pm0.58$     & $<6.2$               \\
              & Si                     & NAA                       & -                       & \ccsg$<6.4\cdot10^{-5}$ & \ccsg $<6.4\cdot10^{-4}$ & -                  & -                 & -                    \\
  cables      & Haefele 10mil          & $\gamma$ spec.            & $<230$                  & $210\pm60$              & $<150$                   & $60\pm30$          & $3.00\pm0.60$     & $<10$                \\
              & Haefele 2mil           & $\gamma$ spec.            & $<320$                  & $780\pm170$             & $<380$                   & $<270$             & $3.5\pm1.7$       & $<35$                \\
              & Tecnomec 3mil          & $\gamma$ spec.            & $<73$                   & $<49$                   & $<060$                   & $<10$              & $1.43\pm0.39$     & $<5.7$               \\
              & Tecnomec 2mil          &                           & $<42$                   & $<29$                   & $<35$                    & $<6$               & $0.83\pm0.23$     & $<3.3$               \\
  \rclb \ccw  & 4 above  \ccw          & \ccw                      & $<660$                  & $990\pm310$             & $<620$                   & $60\pm310$         & $8.7\pm3.0$       & $<54$                \\
  front end   & \ccw CC3               & \ccw $\gamma$ spec.       & $760\pm380$             & $2670\pm380$            & $<1300$                  & $<300$             & $13.3\pm3.8$      & $<21$                \\
  mini-       & coated                 & ICPMS                     & $10.8\pm3.2$            & -                       & $10.8\pm3.2$             & -                  & \ccsg$<0.11$      & $0.0201\pm0.0060$    \\
  shroud      & glued                  & ICPMS                     & $7.2\pm2.2$             & -                       & $7.2\pm2.2$              & -                  & \ccsg$>1.7$       & $0.0214\pm0.0064$    \\
              & 2 above                &                           & \ccsg$18\pm5$           & -                       & \ccsg$18\pm5$            & -                  & $>1.7$            & \ccsg$0.043\pm0.013$ \\
  LAr veto    & fibers                 & ICPMS                     & $44\pm4$                & -                       & $44\pm4$                 & -                  & $0.350\pm0.070$   & $0.032\pm0.003$      \\
              & copper supports        &                           & $130\pm130$             & $520\pm520$             & $130\pm130$              & -                  & -                 & -                    \\
  \rcsg \ccw  & \ccw2 above            & \ccw                      & $180\pm140$             & $520\pm520$             & $180\pm140$              & \ccw -             & $0.350\pm0.070$   & $0.032\pm0.003$      \\
              & SiPM                   & ICPMS                     & $<1.3$                  & $<3.9$                  & $<1.3$                   & -                  & -                 & -                    \\
              & plastic opt. coupling  & $\gamma$ spec.            & -                       & $<6.1$                  & $4.8\pm0.08$             & -                  & $0.0960\pm0.0096$ & -                    \\
              & Cuflon                 & $\gamma$ spec.            & -                       & $19.5\pm6$              & $12\pm0.5$               & -                  & $0.270\pm0.027$   & -                    \\
              & pins                   & $\gamma$ spec.            & -                       & $141\pm28$              & $<55$                    & -                  & $2.07\pm0.20$     & -                    \\
              & screws                 & $\gamma$ spec.            & -                       & $191\pm51$              & $<74$                    & -                  & $<1.5$            & -                    \\
              & glue EP601             & $\gamma$ spec.            & $<6$                    & $<2.2$                  & $<6$                     & -                  & $<0.052$          & -                    \\
  \rcsg \ccw  & \ccw 6 above           &  \ccw                     & $<7.3$                  & $351\pm97$              & $16.8\pm140$             & \ccw -             & $2.4\pm1.8$       & \ccw -               \\
              & PMT top                & $\gamma$ spec.            & -                       & $<15\cdot10^3$          & $<18\cdot10^3$           & -                  & $<82$             & -                    \\
              & PMT bottom             & $\gamma$ spec.            & -                       & $<12\cdot10^3$          & $<14\cdot10^3$           & -                  & $<64$             & -                    \\
              & voltage dividers (VD)  & $\gamma$ spec.            & -                       & $<18\cdot10^3$          & $<8\cdot10^3$            & -                  & $<180$            & -                    \\
  \rcsg \ccw  & \ccw PMT top + 9xVD    &  \ccw                     & \ccw -                  & $<26\cdot10^3$          & $<22\cdot10^3$           & \ccw -             & $<190$            & \ccw -               \\
  \rcsg \ccw  & \ccw PMT bottom + 7xVD &  \ccw                     & \ccw -                  & $<20\cdot10^3$          & $<17\cdot10^3$           & \ccw -             & $<140$            & \ccw -               \\
              & copper shrouds         & ICPMS                     & $62\pm6$                & $250\pm25$              & $62\pm6$                 & -                  & -                 & -                    \\
              & tetratex               & $\gamma$ spec.            & -                       & $282\pm28$              & $132\pm13$               & -                  & $18.4\pm1.8$      & -                    \\
  \rcsg \ccw  & \ccw 2 above           & \ccw                      & $62\pm6$                & $532\pm53$              & $194\pm19$               & \ccw -             & $18.4\pm1.8$      & \ccw -               \\
  \bottomrule
\end{tabular}

\end{sidewaystable}

% vim: tw=90
