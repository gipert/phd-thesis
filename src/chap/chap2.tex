\chapter{The \gerda\ experiment}\label{chap:gerda}

The \gerda\ experiment has been proposed in 2004~\cite{gerda-proposal} to search for
neutrinoless double-beta decay with high-purity germanium detectors enriched in the
\gesix\ double-beta emitter. The proposal lies in the path opened by the Heidelberg-Moscow
(\hdm)~\cite{Klapdor2001} and \igex~\cite{Aalseth2002} experiments, aiming to develop the
germanium technology towards large-scale, \bkgfree\ experimental conditions that
could tackle the scale of $\mathcal{O}(10^{26})$~yr sensitivity on the neutrinoless
double-beta decay half-life.

\marginnote{\gerda\ \\ phases}
Since the start of the data taking in 2008 the experiment, located in hall~A of the Gran
Sasso National Laboratories (LNGS) in Italy, has been running through two distinct
experimental phases (\phaseone\ and \phasetwo) and finally reached the target total
\bkgfree\ exposure of 100~\kgyr\ in December 2019. Detectors from the former \hdm\ and
\igex\ experiments (of semi-coaxial geometry) along with newly produced diodes (of BEGe
geometry type, shorthand for Broad Energy Germanium detectors) were deployed bare into
liquid argon (LAr) during \phaseone, following a suggestion by \fillme, for a total amount
of 21.3~kg of germanium. \phaseone\ ended in June 2013 with a total exposure of
21.6~\kgyr\ and a background index in the region of interest of \pIbi~\cite{Agostini2016}.
Short after the upgrade works for \gerda\ \phasetwo\ started: a new event veto system
based on the LAr scintillation light was installed along with an additional 20~kg of
BEGe-type detectors.  The newly designed veto system allowed for a significant reduction
of the background index (BI) down to the \vctsper\ scale, allowing \gerda\ to seamlessly
run in \bkgfree\ conditions for its full second experimental phase and surpass the
\powtenyr{26} threshold sensitivity~\cite{Agostini2019} in April 2018. Data taking was
then stopped again to permit a third hardware upgrade, during which another \fillme~kg of
enriched germanium in the form of four inverted-coaxial geometry detectors were deployed
and the LAr veto system improved in detection efficiency in the space between the
detectors. This last part of \phasetwo, which will be referred as \phasetwop in the
following, ended in December 2019 after collecting the total exposure of \fillme~\kgyr\
and establishing the final \gerda\ upper limit on the neutrinoless double-beta decay
half-life of \gerdafinallimit.

The \phasetwop\ upgrade is partly also a test bench for the next-generation successor of
\gerda\ in the field of double-beta decay physics with \gesix, the LEGEND experiment. The
collaboration, formed in October 2016 from the \gerda\ and \majorana~\cite{Abgrall2014}
(the latter experiment searching for \onbb\ with germanium detectors in \fillme, USA),
pursues the goal of building a tonne-scale \gesix\ experiment and reaching the
$\mathcal{O}(10^{27})$ sensitivity scale.

% vim: tw=90
