%!TEX root = ../main.tex

\chapter{The \gerda\ experiment}\label{chap:gerda}

The GERmanium Detector Array (\gerda) experiment has been proposed in
2004~\cite{gerda-proposal} to search for neutrinoless double-beta decay with High-Purity
Germanium detectors (HPGe) enriched in the \gesix\ double-beta emitter. The proposal lies
in the path opened by the Heidelberg-Moscow (\hdm)~\cite{Klapdor2001} and
\igex~\cite{Aalseth2002} experiments, aiming to develop the germanium technology towards
large-scale, background-free experimental conditions that could tackle the scale of
$\mathcal{O}(10^{26})$~yr sensitivity on the neutrinoless double-beta decay half-life. The
history of the achievements in \thalfzero\ limit setting and background level with \gesix\
is presented in \cref{img:exp:ge76-history}, top. On the bottom plot, a zoom in the
\gerda\ data taking period showing the experimental progresses in terms of \thalfzero\
sensitivity, lower limit and collected active exposure.  \gerda\ data taking officially
ended in December 2019 after hitting the target total background-free exposure of 100~\kgyr\ and
establishing itself as the leading experiment in the field in terms of lowest background
level ever achieved around \qbb~\cite{Agostini2019a}.

\blocktitle{\gerda\ \\ phases}
Since the start of the data taking in 2008 the experiment, located in hall~A of the Gran
Sasso National Laboratories (LNGS) in Italy, has been running through two distinct
experimental phases (\phaseone\ and \phasetwo). Detectors from the former \hdm\ and \igex\
experiments (of semi-coaxial geometry) along with newly produced diodes (of \bege\
geometry type, shorthand for Broad Energy Germanium detectors) were deployed bare into
liquid argon (LAr) during \phaseone, following a suggestion by ref.~\cite{Heusser1995},
for a total amount of 21.3~kg of germanium. \phaseone\ ended in June 2013 with a total
exposure of 21.6~\kgyr\ and a background index in the region of interest of
\pIbi~\cite{Agostini2016}.  Short after the upgrade works for \gerda\ \phasetwo\ started:
a new event veto system based on the LAr scintillation light was installed along with an
additional 20~kg of \bege-type detectors.  The newly designed veto system allowed for a
significant reduction of the background index (BI) down to the \powctsper{-4} scale, allowing
\gerda\ to seamlessly run in background-free conditions for its full second experimental phase
and surpass the \powtenyr{26} sensitivity threshold in April 2018~\cite{Agostini2019a}.
Data taking was then stopped again to permit a third hardware upgrade, during which
another 9.6~kg of enriched germanium in the form of five inverted-coaxial geometry
detectors was deployed\footnote{Note however that a semi-coaxial detector (\ANG{1}) and
the natural \GTF{} detectors were removed.}. Moreover, the LAr veto system was exchanged
with a more efficient one, thanks to the denser fiber curtain and the addition of a shroud
enclosing the central string, now constituted by inverted-coaxial detectors. This last
part of \phasetwo, which will be referred as \phasetwop\ in the following, ended in
December 2019 after collecting the total exposure of \fillme{fillme}~\kgyr\ and
establishing the final \gerda\ upper limit on the neutrinoless double-beta decay half-life
of \gerdafinallimit. When generally referring to the \phasetwo\ period, if not specified
in the following, \phasetwop\ must be considered as implicitly included.
\begin{figure}
  \centering
  \includegraphics{plots/0nbb-ge76-history.pdf}
  \includegraphics{plots/gerda-history.pdf}
  \caption{%
    Top: history of lower limits (68\% CL, 90\% CL since 1991) and background indices from
    \gesix\ neutrinoless double-beta decay experiments, extracted from~\cite{Fiorini1967,
    Bellotti1984, Caldwell1989, Reusser1991, Vasenko1989, Klapdor2001, Aalseth2002,
    Klapdor2004, Alvis2019, gerda-final} and references therein. Original design by Karl-Tasso
    Kn\"opfle.  Numbers attached to single data points correspond to exposures in
    mol(\gesix)$\cdot$yr. Bottom: evolution and milestones set by the \gerda\ experiment
    through its three experimental phases. The exposure gain, the neutrinoless double-beta
    decay half-life sensitivities and upper limits are tracked over time. \fillme{to be
    updated}
  }\label{img:exp:ge76-history}
\end{figure}
\newpar
The \phasetwop\ upgrade is partly also a test bench for the next-generation successor of
\gerda\ in the field of double-beta decay physics with \gesix, the LEGEND experiment. The
collaboration, formed in October 2016 from the \gerda\ and \majorana~\cite{Abgrall2014}
(the latter experiment searching for \onbb\ with germanium detectors at the Sanford
Underground Research Facility (SURF) in USA), pursues the goal of building a tonne-scale
\gesix\ experiment and reaching the $\mathcal{O}(10^{28})$~yr sensitivity scale. The first
phase of the experiment, LEGEND-200, deploys 200~kg of germanium in the existing \gerda\
infrastructure at LNGS and it is currently in commissioning phase.
\newpar
As the present thesis work focuses on \gerda\ \phasetwo\ data, the description of the
experimental setup and the main analysis techniques given in the following will be limited
to that time period. The chapter is structured as follows. In \cref{sec:gerda:setup} a
general overview of the \gerda\ \phasetwo\ (and \phasetwop) apparatus is given. In
\cref{sec:gerda:cuts} the working principles of the main background reduction techniques
that allow \gerda\ to operate in the background-free regime are outlined. The application
of these event-selection criteria to the \phasetwo\ data is then described in
\cref{sec:gerda:ana}, and the final results are presented.

\section{Overview of the Phase {\normalfont\textit{II}} experimental setup}%
\label{sec:gerda:setup}

The \gerda\ experiment is located in hall~A of the LNGS laboratories, at a depth of about
3500~m water equivalent, to suppress cosmogenically-induced background
sources~\cite{Wiesinger2018}. The germanium detectors are arranged into strings within a
cryostat filled with 64~m$^3$ of liquid argon (LAr), which acts as a shielding and cooling
medium at the same time. The cryostat itself is enclosed by a large tank containing
590~m$^3$ of ultra-pure water.  Besides the additional shielding effect, this water layer
act as a medium for a \v{C}erenkov veto system with 66 photomultiplier tubes (PMTs)
against muons. An array of scintillating panels is installed on the top of the clean room
to complete the muon veto system~\cite{Freund2016}. A simplified representation of the
experimental setup is given in \cref{fig:setup:overview}, together with a picture taken
from the outside.

\begin{figure}
  \centering
  \includegraphics[width=\textwidth]{setup/gerda-overview.pdf}
  \caption{%
    On the right: artist view of the \gerda\ experimental setup. On the left: a picture of
    taken during the inauguration in November 2010. The experiment, installed in hall A of
    the Gran Sasso National Laboratories in Italy, deploys an array of germanium detectors
    enriched in \gesix\ bare in liquid argon, together with a liquid argon scintillation
    light veto system. The cryostat is submerged in a water tank to provide additional
    shielding from external background sources. A plastic scintillating panel system is
    installed on the top of the whole structure as an active muon veto, together with the
    water tank.
  }\label{fig:setup:overview}
\end{figure}

\blocktitle{detectors}
The \gerda\ \phasetwo\ array is organized in 7 vertical strings, holding 40 detectors in
total. The detectors can be divided in three groups: the \bege\ detectors, the
semi-coaxial \m{ANG} and \m{RG}, and the semi-coaxial \m{GTF} detectors. The detectors of
the first two groups are made of germanium enriched in \gesix, the third group includes
detectors with natural isotopic germanium abundance. During the upgrade works for
\phasetwop\ in 2018 four enriched inverted-coaxial \IC{} detectors were introduced to
replace the three \GTF{} detectors in the central string, for a total of 41 detectors
deployed.
\newpar
All \gerda\ HPGe detectors are made of high-purity p-type germanium, which is initially
used to pull crystals, typically fuse-shaped (see for example fig.~4.1a in
\cite{Yonenaga2019}).  Crystals are then cut in slices, and each of them is further
processed to obtain the final detector geometry. The electrodes for signal read-out and
voltage biasing are then fabricated on the detector surface. The \nplus\ contact, where the
external voltage is applied, `wraps around' the detector. It is obtained by deposition of
a lithium layer on the surface, which diffuses below the surface until a depth of
$\sim$1~mm during the subsequent thermal annealing cycles. The presence of lithium
impurities effectively creates a region with decreased charge collection efficiency (CCE),
or `dead-layer', even when biased at full-depletion voltages.  In this region, the CCE is
zero at the surface and reaches its maximal value at the full charge collection depth
(FCCD). The \pplus\ electrode, where the signal is read out, is instead fabricated by boron
implantation, and the dead layer it produces is typically smaller, at the level of
hundreds of microns. The two conductive surfaces are separated by an insulating region,
which is typically produced by excavating a `groove'. In some cases such groove is
passivated by deposition of a germanium-oxide layer.

\blocktitle{%
  \includegraphics[width=15mm]{gedet/BEGe.png}\\
  \includegraphics[width=15mm]{gedet/SemiCoax.png}\\
  \vspace{2mm} % tweak
  \includegraphics[width=15mm]{gedet/InvCoax.png}%
}
The \gerda\ \phasetwo\ detectors before the 2018 upgrade can be classified according to
two different geometry types: semi-coaxial and \bege. In the semi-coaxial design, a
bore-hole is excavated along the central axis to accomodate the \pplus\ electrode. With
such a configuration, relatively large detector masses can be achieved, of the order of
2--3~kg. The \ANG{} (5), \RG{} (2) and \GTF{} (3) detectors, inherited from the \hdm\ and
\igex\ experiments and already used in \phaseone, are of the semi-coaxial type. Their
total mass amounts to 23.2~kg of germanium, while the enrichment fractions are in the
85.5--88.3\% range. For \phasetwo, 20~kg of germanium enriched at 87.8\% was procured by
the \gerda\ collaboration for the production of 30 new diodes of the \bege\ type. The
Broad Energy Germanium detector design does not include a bore-hole, therefore the \pplus\
contact is a small, dot-shaped surface at the center of one of the two detector sides. The
absence of a bore-hole makes this kind of detectors harder to fully deplete, requiring
lower impurity concentrations and smaller masses, generally lower than 1~kg. A detailed
description of the characteristics of the \bege\ detectors, from germanium procurement to
diode production can be found in~\cite{Agostini2015e, Agostini2018a, Agostini2019}. For
\phasetwop\ four new enriched \IC{} detectors of the inverted-coaxial type were fabricated
and deployed in place of the natural \GTF{} detectors. This new inverted-coaxial geometry
design includes a dot-shaped \pplus\ contact, to enable the \bege-like pulse-shape
discrimination features, and a bore-hole of the other side, to make it possible to achieve
large detector masses. Details about the production and characterization of the  four
inverted-coaxial detectors deployed in \gerda\ \phasetwop\ can be found
in~\cite{inverted-paper}.

\begin{figure}
  \centering
  \includegraphics[height=7cm]{gedet/phII-array.png}
  \hspace{0.5cm}
  \includegraphics[width=4cm, trim=0 -4cm 0 0]{gedet/phII-array-calib.png}
  \hspace{0.5cm}
  \includegraphics[height=7cm]{gedet/phIIp-array.png}
  \caption{%
    The \gerda\ \phasetwo\ detector array. On the left: setup from the start of Phase II
    (December 2015). On the right: \phasetwop\ setup after the 2018 upgrade works. The
    main difference between the two configurations is the presence of upside-down detectors
    in the first configuration and the inverted-coaxial detectors in place of the natural
    detectors in the central string in the \phasetwop\ configuration. In the center: top
    view of the \phasetwo\ array, with the three calibration sources. \fillme{different
    colors for detector types}
  }\label{fig:setup:array}
\end{figure}

\blocktitle{array \\ instrumentation}
As already mentioned, the \gerda\ \phasetwo\ detectors are arranged into 7 strings, packed
closely together as depicted in \cref{fig:setup:array}, to maximize the
multi-detector event rejection efficiency. Since the main background sources in
\phaseone\ were located close to the detectors, the design of the mounting and cabling
system has been carefully chosen to minimize the mass. The detector holder unit consists
of a low-mass, intrinsically radio-pure silicon plate and three vertical copper bars to
take the detector weight and connect the modules between themselves within a string. The
silicon plate provides the substrate onto which signal and high voltage cables are
attached. The Ge detectors are read out with custom-produced, cryogenic and low
radioactivity preamplifiers called `CC3'~\cite{Riboldi2015}. The Ge readout electrode is
connected to the JFET-PCB by a flexible flat cable. Two different cable types are adopted
for the signal and HV contact: the HV cables are made from 10 mils Cuflon\reg, or 3 mils
Pyralux\reg, the signal FFCs from 3 mils Cuflon\reg\ or Pyralux\reg. See
\cref{fig:setup:magevolumes}a. \fillme{PhaseII+ changes?}

\blocktitle{LAr veto}
To improve the sensitivity on the \onbb\ half-life and operate in the background-free regime, an
additional active veto system to collect the LAr scintillation light produced by
background events was designed and installed during the upgrade works for \phasetwo. A
cylindrical hybrid design was chosen to detect the light information: a curtain made of
light-guiding plastic fibers coupled to a ring of silicon photomultipliers (SiPMs) to
surround the array and PMTs on the top (9) and on the bottom (7) (see
\cref{fig:setup:magevolumes}d). To enhance the light collection efficiency two copper
shrouds (visible in \cref{fig:setup:magevolumes}d) coated with a reflective Tetratex\reg\
layer were added between the fiber shroud and the PMT holder plates. The latter were
coated with a reflective VM2000 layer. Another light collection improvement introduced by
the \phasetwo\ upgrade is the installation of nylon (mini-)shrouds enclosing each detector
string (\cref{fig:setup:magevolumes}b). The presence of these shrouds provides an
essential mechanical barrier to reduce the background from \kvz\ ions naturally present in
LAr, which undergo \b-decay and can mimic the \onbb\ signature at \qbb. Being made of
transparent nylon material, in contrast to the ones from \phaseone\ made of copper, the
mini-shrouds let the light propagate more efficiently to a close-by light collecting
surface. To match the fibers and PMTs spectral response many surfaces in the close vicinity
of the array were coated with tetraphenil-butadiene (TPB), a wavelength shifting material.
Coating has been applied on mini-shrouds, fiber-shroud, copper shroud, PMTs as well as
their holder plates. The reader is referred to ref.~\cite{Agostini2018a} for the detailed
LAr veto instrumentation technical specifications, as implemented for the first part of
\phasetwo.
\newpar
The LAr veto system was upgraded in 2018 for \phasetwop\ to achieve a higher vetoing
efficiency. The fiber shroud was exchanged and its fiber density increased by 50\%. A new
fiber curtain was fabricated to wrap around the central string and enhance the detection
probability in volumes close to the detectors (see \cref{fig:setup:pictures} and
\cref{fig:setup:magevolumes}c). The light collected by the fibers is read out by two SiPM
arrays at the top end.

\begin{figure}
  \includegraphics{setup/pic-collage.pdf}
  \caption{%
    Various pictures of the \gerda\ \phasetwo\ setup, taken during the upgrade
    works.
  }\label{fig:setup:pictures}
\end{figure}

\blocktitle{calibration \\ system}
The \gerda\ weekly calibrations are performed by lowering three \Th\ sources into LAr in
the close vicinity of the array, at the same radial distance from the array central axis
and evenly spaced. Each source, when lowered, just fits into the space between the
cylinder of the LAr veto system and two neighboring outer strings of the detector array,
thereby the sources enter the inner volume of the LAr veto system by three slots in the
top PMT plate.  Three sources were produced and characterized for the first part of
\phasetwo~\cite{Baudis2015} and then again for \phasetwop. The LAr veto instrumentation is
usually switched off during calibration runs because of the too high source activity of
$\mathcal{O}(10)$~kBq.  However, less intense \Ra\ sources are also available and can be
easily exchanged with the standard ones. Special calibration data has been acquired with
these sources and the LAr light instrumentation turned on, to study the performance of the
LAr veto system. The calibration of the experimental setup is extensively described
in~\cite{calib-paper}.

\blocktitle{data \\ acquisition}
A FADC system records traces from germanium detectors (40), PMTs (16) and SiPMs (15) of
the LAr veto, PMTs and scintillating panels of the muon veto when an energy deposition
greater than about 100~keV occurs in at least one of the germanium detectors\footnote{The
  exact trigger threshold is detector- and run-dependent and varies between 20~keV and
200~keV.}. Besides of real physical triggers, two special artificial events are recorded
by the DAQ: test signals injected with a pulser in each germanium detector and baseline
events with no physical trigger to study the electronic noise. These events are recorded
at fixed time intervals during data taking.
\newpar
The energy deposition associated to each germanium detector signal is
determined via a Zero Area Cusp (ZAC) filter which is optimized off-line for each detector
and each calibration run~\cite{Agostini2015}. PMT and SiPM hits are reconstructed in the
offline analysis following the procedure documented in~\cite{Agostini2018a}. Each event
has to pass a series of quality cuts tailored to discard unphysical events with very high
efficiency (see \cref{sec:gerda:cuts}). The reconstructed trigger positions are converted
into time differences relative to the first trigger found in the germanium detector
traces. Trigger positions and amplitudes are subsequently used together with hits from the
SiPM to test the LAr veto condition. The algorithms were implemented in the \gelatio\
framework~\cite{Agostini2011} which is used to process \gerda\ data. Each event is
characterized by the calibrated energy deposited in the Ge diode, a data quality flag, the
classification as signal or background event from the pulse shape analysis, and veto flags
from the muon veto and LAr veto systems.

\begin{figure}
  \centering
  \includegraphics[width=\textwidth]{setup/mage-volumes.pdf}
  \caption{%
    Implementation of the \gerda\ array in \mage, visualized using the \geant\
    visualization drivers. From left to right: \emph{a)} the \phasetwo\ holder mounting,
    composed of silicon plates and copper bars, and the high-voltage and signal flex flat
    cables.  Front-end electronics are on the top end, \emph{b)} the full \phasetwo\ array
    instrumentation, including the transparent nylon mini-shrouds, \emph{c)} the full
    \phasetwop\ array instrumentation, including the central fiber shroud (in green),
    \emph{d)} the full \phasetwo\ LAr veto system, including the outer fiber shroud, the
    Tetratex\reg-coated copper shrouds (above and below the fibers) and the two PMT
    arrays, \emph{e)} the \phasetwop\ LAr veto system without the copper shrouds.
  }\label{fig:setup:magevolumes}
\end{figure}

\section{Background reduction techniques}%
\label{sec:gerda:cuts}

\begin{figure}
  \centering
  \includegraphics[width=\textwidth]{gedet/gerda-events.png}
  \caption{%
    Signal and background events in \gerda, working principles of the main background
    reduction techniques. From left to right: \emph{signal-like events}: the point-like
    topology in dense detectors of double-beta decays generate distinct single-detector
    pulse shapes. \emph{granularity cut}: external, background \g{}s can deposit energy
    in multiple detectors. \emph{pulse-shape discrimination}: insights on the event
    topology can be obtained by analyzing its waveform. Single-site events, multi-site
    events, \b{}s and \a{}s on the surface can be discriminated with offline algorithms
    depending on the specific detector geometry. \emph{LAr veto}: background events that
    deposit energy in germanium and LAr at the same time can be efficiently vetoed by
    the \gerda\ LAr veto.
  }\label{fig:gerda:event-types}
\end{figure}

Various background mitigation techniques are adopted, both at the data acquisition level
(online) and the analysis level (offline) in \gerda\ to lower the background index to the
background-free level of \pIIbi. The techniques outlined in the following have been
gradually developed and refined during several years of research and publications, and
have been employed for the \phasetwo\ final (re-)analysis in~\cite{gerda-final}.
Documentation about partial analyses of the \gerda\ \phasetwo\ data published
in~\cite{Agostini2015a, Agostini2017, Agostini2018, Agostini2019a} can be found in those
publications and references therein.

\blocktitle{muon veto}
Muons may cause a substantial background to rare event searches like \gerda\ by generating
counts at \qbb\ either through direct energy deposition in the detectors or through
e.g.~decay radiation of spallation products. At LNGS the cosmic muon flux is reduced by a
factor of ${\sim}10^6$ to a rate of ${\sim}3.4 \cdot 10^{−4}~\text{s}^{-1}\text{m}^{-2}$,
which is sill sufficient to generate a non-negligible background of the order of
\powctsper{-3}.  As already described in \cref{sec:gerda:setup}, a muon veto comprising of
a water \v{C}erenkov veto and a scintillator veto was implemented in \gerda\ to reduce
this background contribution. An event with energy deposition in germanium is flagged
as muon-induced background if a coincidence with the muon veto signal occurs in a $\pm
10$~\mus\ window around the germanium trigger. The efficiency of the muon veto system
has been estimated to be of ${\sim}99$\%, leading to a residual background index of
${\sim}$\powctsper{-5}~\cite{Freund2016}.

\blocktitle{LAr veto}
The primary role of liquid argon in \gerda\ is to keep the germanium detectors at a
cryogenic operational temperature and provide a passive shielding medium against external
backgrounds. Moreover, the LAr can be also employed as a detector medium in an active veto
system, thanks to its scintillation properties. The production mechanism of the
scintillation light in LAr is known since several decades and is deeply described in
literature and its spectrum is today well known. The incident particles deposit their
energy mainly by interactions with the electron shell of the argon atoms which leads to
either an excitation or an ionization of argon atoms. Excited argon atoms are frequently
called `excited dimers' or `excimers' in the literature. Their decay is accompanied by the
emission of scintillation light in the vacuum ultraviolet region, whose typical wavelength
is usually cited as $\lambda = 128$~nm~\cite{Heindl2010}. The ratio between excitation and
ionization is strongly dependent on the pressure and density of the argon as well as on
the type of radiation itself. In the case of excitation, the excited argon atom can
directly form an excimer via the collision with neighboring argon atoms. The process is
sketched in \cref{fig:setup:lar-scint}.
\begin{figure}[h]
  \centering
  \includegraphics[width=0.7\linewidth]{lar-scint-mechanism.pdf}
  \caption{%
    Scintillation mechanism of liquid argon (or gaseous argon) via the decay of excited
    dimers. The excited dimer can be either formed directly from an excited argon atom or
    from an ionized atom which forms an ionized dimer before its recombination and the
    following recombination in its molecular form. Drawing courtesy of Christoph
    Wiesinger.
  }\label{fig:setup:lar-scint}
\end{figure}
The excimer itself is meta-stable and appears in two different states: the singlet and the
triplet state~\cite{Jortner1965, McCusker1984}.  The decay of the triplet state is
forbidden due to angular momentum conservation, while the decay of the singlet state is
allowed. Consequently the lifetime of the triplet state is 1.59~\mus\ which is
significantly higher than the 6~ns of the singlet state. The scintillation light yield
(combined for both components) is roughly 40 photons/keV, measured in ultra-pure
LAr~\cite{Doke1988}. This value is dependent on different factors, like the presence of
contaminants, the pressure and density of the argon as well as the ionization density of
the incident particle~\cite{Doke1988}.
\newpar
The goal of the \gerda\ LAr veto is to reject those types of background events in the
germanium detectors that simultaneously deposit energy in the surrounding LAr, and hence
generate scintillation. These background types mainly include \g-ray background from Ra
and Th decays in solid materials inside and around the detectors. But also other types of
background can successfully be rejected, such as muons or decays from \Arh\ or \kvz. An
event depositing energy in the germanium detectors is discarded as background if a
coincidence with the LAr veto signal is found in the time window spanned by the germanium
traces.  Since the lifetime of the LAr triplet state significantly depends on the argon
purity~\cite{Amsler2007}, it is possible to monitor the purity of LAr over time.
\cref{fig:lar:triplet-lifetime} shows the lifetime values measured about every month since
the start of \phasetwo. The average measured lifetime dropped after the \phasetwop\
upgrade works from $\sim$1~\mus\ to $\sim$0.9~\mus, as a probable consequence of
maintenance works of the cryogenic system.
\begin{figure}
  \centering
  \includegraphics{plots/lar/lar-triplet.pdf}
  \caption{%
    LAr triplet lifetime regularly measured during \phasetwo. The decrease of the LAr
    purity in 2018 might be attributed to maintenance works of the cryogenic
    infrastructure.
  }\label{fig:lar:triplet-lifetime}
\end{figure}
The veto condition is realized when the signal in at least one channel (SiPM array or PMT)
exceeds a certain threshold (around one photo-electron or less) within a certain time
window around the germanium trigger (usually few microseconds). The \onbb-signal
efficiency of the LAr veto cut can be estimated by evaluating the number of test pulses
and baseline events that are randomly flagged as background events. This fraction has been
evaluated to $(97.7 \pm 0.1)$\% and $(98.2 \pm 0.1)$\% for the first part of \phasetwo\
and \phasetwop, respectively. Combining these two estimates for the whole \phasetwo\
results in an efficiency for \onbb\ events of
\[
  \epsilon_\onbbM^\text{LAr-veto} = \fillme{fillme} \;.
\]

\blocktitle{data \\ quality}
Each event has to pass a series of quality cuts tailored to discard unphysical events such
as discharges, pile-up, overflowed events and other problematic traces with very high
efficiency. The \onbb-signal efficiency of the data quality cuts has been estimated by
building an artificial signal-like data sample and applying the data quality cuts to it.
The base for this data sample consists of special pure-baseline events without physical
triggers, which are regularly recorded in \gerda. Since these events are artificially
triggered, no signal is expected in any detector with high probability, and therefore they
can be used to characterize the background noise at a given time during data taking. On
top of these baseline events special averaged waveforms from \Th\ DEP events are added to
baseline events (only one channel per event) to produce the signal-like sample. An
estimation of the acceptance of these events is finally yields:
\[
  \epsilon_\onbbM^\text{QC} = (99.922 \pm 0.002)\% \;.
\]

\blocktitle{pulse-shape \\ discrimination}
The drift of charges created by a ionizing particle in a voltage-biased germanium
detector, which determines the shape of recorded event waveform, depends on the electric
field in the diode. The latter, in particular, depends on the geometry and on crystal
parameters like the impurity concentration and its gradient. Therefore, analysis
techniques can be developed to discriminate between various event types in germanium
detectors. Distinguishing between single-site (SSE) and multi-site (MSE) events is of
primary interest for \gerda, since \onbb\ decays pertain to the SSE class. The two
electrons, in fact, deposit their energy in germanium within 1~mm$^3$ and can be therefore
considered as point-like events. On the other hand, background events caused by
e.g.~multiple Compton scattering of external \g\ rays are mostly of the multi-site type.
Besides MSEs, surface events are another prominent source of background. Energetic \b\
rays created at the \nplus\ electrode surface can penetrate the dead-layer and deposit
energy in the active volume. In particular, the \b\ decay of \kvz, a daughter of \Arh\
naturally present in LAr, is a dangerous background in the ROI because of its high
Q-value. These \b\ decays at the \nplus\ surface can create `slow' pulses with incomplete
charge collection because of the low electric field in the Li-diffused region. The \pplus\
electrode and the insulating groove can be trespassed also by \a\ particles. The
shallowness of the boron implantation of the \pplus\ (hundreds of nanometers) and the
absence of any dead layer in the groove\footnote{The passivation layer, if present, is
usually hundreds of nanometers thick and can be therefore penetrated by \a\ particles.}
let external \b{}s and \a{}s deposit energy in the detector active volume. The intense
electric field causes energy depositions in this region to generate pulses with short
rise times. \a\ events on the \pplus\ electrode are mainly produced by \Po\ accumulated
on its surface, most probably during detector handling. These 5.3~MeV \a\ particles may
lose part of their energy before reaching the active volume and contribute to the
background in the ROI.
\newpar
To mitigate all these background sources, pulse-shape discrimination (PSD) techniques were
developed separately for \bege\ and \scoax\ detectors separately, to be applied after the
LAr veto cut. For the first class a simple univariate cut was sufficient, while for the
latter two techniques were worked out, one based on neural networks and one on the
analysis of the rise time of the pulses.  In order to avoid systematic effects
calibration, training and evaluation of the PSD methods should be performed on pulses with
energies close to these of the expected signal at \qbb.  In practice, appropriate event
sets are extracted from the weekly \Th\ calibration spectra. The PSD methods applied to
the \gerda\ data are briefly outlined in the following. The interested reader is referred
to~\cite{psd-paper} for a detailed treatment of the topic.

\blocktitle{PSD for \\ \bege{}s}
PSD for the \bege\ detectors is based on the \aoe\ ratio, where $A$ is the maximum
amplitude of the current signal and $E$ is the event energy. This technique has been
extensively studied in the past in the context of \gerda~\cite{Agostini2013, Agostini2010,
Budjas2008, Budjas2009, Budjas2009a, Agostini2010}. The motivation in employing such a
relatively simple, univariate cut lies in the observation that in the \bege\ detectors,
thanks to their small \pplus\ contact, the electric field has a special distribution,
resulting in the same shape of pulses induced by drifting holes along paths near the
\pplus\ electrode~\cite{Agostini2010}. Multiple energy depositions in the detector can be
treated as a superposition of several single interactions. It follows that a MSE will have
a lower $A$ compared to a SSE with the same $E$.  A two-sided cut on \aoe\ to cut MSEs
slow pulses and \pplus\ fast events is introduced and determined separately for each
detector. Energy and time stability corrections to \aoe\ are discussed in detail
in~\cite{psd-paper}. As mentioned before, the \aoe\ cut values are determined employing
representative data samples from \Th\ calibration runs.  The low cut position (rejection
of MSEs and slow pulses) is adjusted to achieve a 90\% survival fraction of the
double-escape peak (DEP), a SSE sample. The threshold on the high \aoe\
side (rejection of fast pulses) has been fixed to 3 standard deviations away from the SSE
band distribution. The survival fraction for the \onbb-decay signal has been
calculated assuming that it is the same as for the DEP events. A full analysis of the
statistical and systematic uncertainties yields:
\[
  \epsilon_\onbbM^{A/E} = (87.6 \pm \stat{0.1} \pm \syst{2.6}) \% \;.
\]

\blocktitle{PSD for \\ \scoax{}s}
In semi-coaxial detectors the length of the drift path of the holes depends on the
location of the energy deposition and it induces different types of pulse shapes. Because
of this reason, a simple \aoe\ cut would not be as effective as for the \bege{}s, and
therefore alternative methods have been worked out.
\newpar
The primary method to reject MSE, called here \annmse, consists in a TMVA-based
artificial neural network\footnote{\url{https://root.cern/tmva}} and requires
appropriate selection of input variables (from the rising part of the preamplifier charge
pulse) and training on independent data samples. Several of these samples are available
for training in calibration data (see \fillme{fillme}) and also in physics data (\kvz\
full-energy peak, \nnbb\ events, \a-induced events). The \annmse\ is specifically trained
on \Th\ calibration data, selecting the \Tl\ DEP as a SSE sample and the \Bil\ FEP at
1621~keV as a MSE sample. The classifier cut threshold is then fixed to a 90\% survival
probability for the \Th\ DEP, and the cut signal efficiency is calculated from Monte Carlo
simulations of \onbb-decay events. The obtained \annmse\ signal survival fraction is $(85
\pm 5)$\%.
\newpar
\a-induced events on the \pplus\ electrode surface are rejected by a separate method based
on the analysis of the pulses rise time (RT). These events are generally characterized by
a fast collection time, and their charge collection might be delayed or partial, if
originated in the proximity of the groove. The RT cut exploits the fast rise time of these
\a\ events and is therefore equivalent to a volume cut, which excludes the surfaces
vulnerable to the \a-induced events. The rise time is defined as the time the waveform
needs to reach from 10\% to 90\% of its amplitude. The RT-cut threshold is defined to
maximize the \onbb\ survival fraction and to minimize the signal-to-noise ratio at the
same time through the definition of a figure of merit defined as the product of the \nnbb\
signal survival square-probability and the \a-events rejection probability. The \nnbb\ and
\a-event test samples are obtained from physics data by selecting data in the $[1.0,
1.3]$~MeV and $>3.5$~MeV energy regions, respectively, after \annmse\ and LAr veto cuts.
The \onbb-signal efficiency of the RT cut is assumed to be the same as for the \nnbb\
decays and hence estimated to $(84.3 \pm \stat{0.4} \pm \syst{1.0})$\%. The \annmse\ and
RT-cut efficiency can be combined to obtain an overall survival fraction for the
\onbb-decay events after the combined \annmse-RT cut:
\[
  \epsilon_\onbbM^\text{\annmse-RT} = (64.6 \pm 3.9) \% \;.
\]

\blocktitle{\deltae\ cut}
Events featuring slow charge collection might suffer from ballistic deficit in the ZAC
energy reconstruction~\cite{Agostini2015} and survive the PSD cuts, especially in
semi-coaxial detectors. Therefore an additional rejection criteria is applied based on the
energy reconstructed with different integration times. The figure of merit for such a cut
is the ratio between the energy of an event reconstructed with a short (4~\mus)
integration time $E_\text{s}$ and the energy reconstructed with a long (20~\mus)
integration time $E_\text{l}$. The energy here is reconstructed using a gaussian-shaping
filter, which has been the default \gerda\ energy reconstruction method for \phaseone.
Ballistic deficit is observed to reduce this ratio, therefore the \deltae\ classifier is
defined as $\delta{E} = {[E_\text{s}/E_\text{l}]}_\text{norm} - 1$, where the energy ratio
is normalized to the values assumed by the \Th\ FEP events. The normalization is applied
in a certain calibration validity period and for each detector separately. The cut value
is defined as 3 negative standard deviations away from the mean of the FEP \deltae\
distribution.

\section{Data analysis}%
\label{sec:gerda:ana}

The \phasetwo\ data is presented in this section together with the LAr veto and PSD data.
The statistical analysis used to extract a lower limit for the \onbb\ half-life in \gesix\
is finally presented, and the final results for the combined \gerda\ data are given.

\begin{figure}
  \centering
  \includegraphics{plots/0nbb-results/calib/supercalib.pdf}
  \caption{%
    Top panel: \Th\ calibration summed spectra of \bege, semi-coaxial and inverted-coaxial
    detectors as used to determine the energy resolution curves. Three prominent \Tl\ high
    energy peaks (full-energy, single-escape and double-escape) and the \Bil\
    single-escape peak at 1621~keV are highlighted. Bottom panel: extracted peak widths
    and fitted calibration curves. Points represented by empty markers are excluded from
    the fit because of additional effects that contribute to the width. Triangles label
    peaks broadened by the Doppler effect \fillme{ref?}, diamonds label summation peaks.
    \fillme{fillme}
  }\label{fig:gerda:calib-desc}
\end{figure}

\blocktitle{energy \\ scale and \\ resolution}
As already emphasized in \cref{sec:nbb:exp}, a good energy resolution is a key ingredient
to achieve a high \onbb-decay sensitivity. The main goal of the calibration analysis is
therefore to define and maintain a stable energy scale over years of data taking.  It
is necessary to identify the peak region (and reject all background events with
different energy), combine data from different detectors over extended periods of time,
and efficiently exploit the excellent energy resolution of germanium detectors.
\newpar
As already mentioned in \cref{sec:gerda:setup}, the germanium detectors are calibrated by
exposing them to \Th\ sources with an activity of about 10~kBq. A typical calibration
spectrum is shown in \cref{fig:gerda:calib-desc}. The pattern of \g\ lines in the spectrum
can be exploited to identify certain \g\ lines and calibrate the energy scale of a
detector with their known position in terms of energy.  Additionally, the resolution of a
detector can be determined from the width of the \g\ lines. Once the positions and the
widths of the \g\ lines in an energy spectrum is determined by modeling the peaks with a
suitable analytical function, an function interpolation is performed to obtain the energy
calibration and resolution at other energies. Once these two curves are determined for a
given calibration, they are assumed to be valid until the next one. This validity is
constantly monitored by evaluating the shift of the pulser event energy over time with
respect to its value right after a calibration. Time periods in which in which a detector
shows deviations from its calibration above a certain threshold are removed from the final
analysis in order to meet the stringent requirements for the \onbb\ analysis in terms of
uncertainties on the energy scale and resolution. Fluctuations below this threshold are
taken into account when estimating the systematic contribution to the uncertainty on the
energy resolution. The stability of the energy scale and resolution is also monitored on a
per-calibration basis, and time periods for which detectors show a degraded performance
are excluded from combined analysis data sets.  The calibration spectra that refer to
these combined data sets are obtained by summing together the spectra from all the
calibration runs of the relative time period weighted by their actual validity in time.
Gaussian mixtures are usually not needed to model peaks in these combined spectra, as the
variance of the single centroids and widths is usually small enough to enable the use of
single gaussian distributions with effective parameters. The effective data set energy
resolution is then determined by fitting the square root of a linear function to the
reconstructed \g-line widths. The uncertainty on this effective resolution includes
systematic contributions from the choice of the peak model, the resolution function and
time stability of the experimental setup.  The resolution curves for \bege, semi-coaxial
and inverted-coaxial \phasetwop\ data sets are reported in \cref{fig:gerda:calib-desc} as
an example. The typical energy resolution at \qbb\ is \fillme{fillme} and the associated
uncertainty is of the order of \fillme{fillme}. The calibrated energy spectrum of the
\gerda\ \phasetwo\ data before cuts is shown in \cref{fig:gerda:spectrum-cuts}, empty
histogram. \fillme{update this to final analysis, calibration paper?}

\begin{figure}
  \centering
  \includegraphics[width=0.9\linewidth]{plots/0nbb-results/spectrum-cuts.pdf}
  \caption{%
    Data from the enriched detectors is displayed in a combined spectrum after indicated
    cuts. Main contributions to the spectra are labeled. The insets display the analysis
    window for coaxial and \bege detectors separately, including the background rates
    (solid blue lines). The dashed blue curves depict the 90\% C.L.~limit for a \onbb\
    signal of \gerdafinallimit. \fillme{to be updated}
  }\label{fig:gerda:spectrum-cuts}
\end{figure}

\begin{figure}
  \centering
  \includegraphics[width=0.45\textwidth]{plots/0nbb-results/aoe-on-data.pdf}%
  \includegraphics[width=0.45\textwidth]{plots/0nbb-results/ann-on-data.pdf}
  \includegraphics[width=0.45\textwidth]{plots/0nbb-results/rt-on-data.pdf}
  \caption{%
    The three pulse shape discrimination techniques applied on the \gerda\ \phasetwo\
    data. From left to right: the \aoe\ cut, the \annmse\ cut, the rise-time cut.
    \fillme{to be updated}
  }\label{fig:gerda:psd-on-data}
\end{figure}

\blocktitle{LAr veto and \\ PSD cuts on \\ data}
The first cut applied to \phasetwo\ data (after the quality cuts) is the LAr veto cut.
Event energy histograms before and after this event selection are shown in
\cref{fig:gerda:spectrum-cuts}. The cut clearly suppresses background events from the \Th\
and \Uh\ decay chains, as well as \kvz\ events. The \kvz\ FEP event reduction to 20\% and
18\% in \phasetwo\ and \phasetwop\ data, respectively, demonstrates the effectiveness of
the improved fiber instrumentation installed for \phasetwop. \kvn\ events, which are
characterized by a single \g\ emission at 1461~keV and typically do not deposit energy in
LAr, show a high survival fraction of about 98\%, but cannot enter the ROI at 2039~keV and
are therefore of minor concern. \a\ events that dominate the energy spectrum at higher
energies cannot be vetoed but the LAr instrumentation and still constitute a major
background at \qbb.
\newpar
PSD data versus energy is showed in \cref{fig:gerda:psd-on-data} subdivided according to
the PSD method. Data from \bege\ and inverted-coaxial detectors, for which the \aoe\ cut
is used, is shown in the top-left plot. Data from semi-coaxial detectors, for which the
\annmse\ and the rise-time cuts are implemented, is shown in the remaining plots. Colored
data points correspond to events that survive the PSD cut.

\blocktitle{the ROI}
Events in the ROI are also completely cut because fuck it.

% vim: tw=90
