%!TEX root = ../main.tex

\chapter{The \gerda\ experiment}\label{chap:gerda}

The GERmanium Detector Array (\gerda) experiment has been proposed in
2004~\cite{gerda-proposal} to search for neutrinoless double-beta decay with High-Purity
Germanium detectors (HPGe) enriched in the \gesix\ double-beta emitter. The proposal lies in the
path opened by the Heidelberg-Moscow (\hdm)~\cite{Klapdor2001} and
\igex~\cite{Aalseth2002} experiments, aiming to develop the germanium technology towards
large-scale, \bkgfree\ experimental conditions that could tackle the scale of
$\mathcal{O}(10^{26})$~yr sensitivity on the neutrinoless double-beta decay half-life. The
\gerda\ data taking officially ended in December 2019 after hitting the target total
\bkgfree\ exposure of 100~\kgyr\ and establishing itself as the leading experiment in the
field in terms of lowest background level ever achieved around \qbb~\cite{Agostini2019a}.

\partitle{\gerda\ \\ phases}
Since the start of the data taking in 2008 the experiment, located in hall~A of the Gran
Sasso National Laboratories (LNGS) in Italy, has been running through two distinct
experimental phases (\phaseone\ and \phasetwo). Detectors from the former \hdm\ and \igex\
experiments (of semi-coaxial geometry) along with newly produced diodes (of \bege\
geometry type, shorthand for Broad Energy Germanium detectors) were deployed bare into
liquid argon (LAr) during \phaseone, following a suggestion by ref.~\cite{Heusser1995},
for a total amount of 21.3~kg of germanium. \phaseone\ ended in June 2013 with a total
exposure of 21.6~\kgyr\ and a background index in the region of interest of
\pIbi~\cite{Agostini2016}.  Short after the upgrade works for \gerda\ \phasetwo\ started:
a new event veto system based on the LAr scintillation light was installed along with an
additional 20~kg of \bege-type detectors.  The newly designed veto system allowed for a
significant reduction of the background index (BI) down to the \powctsper{-4} scale, allowing
\gerda\ to seamlessly run in \bkgfree\ conditions for its full second experimental phase
and surpass the \powtenyr{26} sensitivity threshold in April 2018~\cite{Agostini2019a}.
Data taking was then stopped again to permit a third hardware upgrade, during which
another 9.6~kg of enriched germanium in the form of five inverted-coaxial geometry
detectors was deployed\footnote{Note however that a semi-coaxial detector (\ANG{1}) and
the natural \GTF{} detectors were removed.}. Moreover, the LAr veto system was exchanged
with a more efficient one, thanks to the denser fiber curtain and the addition of a shroud
enclosing the central string, now constituted by inverted-coaxial detectors. This last
part of \phasetwo, which will be referred as \phasetwop\ in the following, ended in
December 2019 after collecting the total exposure of \fillme{fillme}~\kgyr\ and
establishing the final \gerda\ upper limit on the neutrinoless double-beta decay half-life
of \gerdafinallimit. With that said, the ``\phasetwo'' period includes \phasetwop\ in the
following.

\begin{figure}
  \centering
  \includegraphics{plots/0nbb-ge76-history.pdf}
  \includegraphics{plots/gerda-history.pdf}
  \caption{%
    Evolution and milestones set by the \gerda\ experiment through its three experimental
    phases. The background index, the exposure gain, the neutrinoless double-beta decay
    half-life sensitivities and upper limits are tracked in time.
  }
\end{figure}

The \phasetwop\ upgrade is partly also a test bench for the next-generation successor of
\gerda\ in the field of double-beta decay physics with \gesix, the LEGEND experiment. The
collaboration, formed in October 2016 from the \gerda\ and \majorana~\cite{Abgrall2014}
(the latter experiment searching for \onbb\ with germanium detectors at the Sanford
Underground Research Facility (SURF) in USA), pursues the goal of building a tonne-scale
\gesix\ experiment and reaching the $\mathcal{O}(10^{28})$~yr sensitivity scale. The first
phase of the experiment, LEGEND-200, deploys 200~kg of germanium in the existing \gerda\
infrastructure at LNGS and it is currently in commissioning phase.

As the present thesis work focuses on \gerda\ \phasetwo\ data, the description of the
experimental setup and the main analysis techniques given in the following will be limited
to that time period. The chapter is structured as follows. In \cref{sec:gerda:setup} a
general overview of the apparatus is given \fillme{expand}.

\section{Overview of the Phase {\normalfont\textit{II}} experimental setup}%
\label{sec:gerda:setup}

The \gerda\ experiment is located in hall~A of the LNGS laboratories, at a depth of about
3500~m water equivalent, to suppress cosmogenically-induced background
sources~\cite{Wiesinger2018}. The germanium detectors are arranged into strings within a
cryostat filled with 64~m$^3$ of liquid argon (LAr), which acts as a shielding and cooling
medium at the same time. The cryostat itself is enclosed by a large tank containing
590~m$^3$ of ultra-pure water.  Besides the additional shielding effect, this water layer
act as a medium for a \v{C}erenkov veto system with 66 photomultiplier tubes (PMTs)
against muons. An array of scintillating panels is installed on the top of the clean room
to complete the muon veto system~\cite{Freund2016}. A simplified representation of the
experimental setup is given in \cref{fig:setup:overview}, together with a picture taken
from the outside.

\begin{figure}
  \centering
  \includegraphics[width=\textwidth]{setup/gerda-overview.pdf}
  \caption{%
    On the right: artist view of the \gerda\ experimental setup. On the left: a picture of
    taken during the inauguration in November 2010. The experiment, installed in hall A of
    the Gran Sasso National Laboratories in Italy, deploys an array of germanium detectors
    enriched in \gesix\ bare in liquid argon, together with a liquid argon scintillation
    light veto system. The cryostat is submerged in a water tank to provide additional
    shielding from external background sources. A plastic scintillating panel system is
    installed on the top of the whole structure as an active muon veto, together with the
    water tank.
  }\label{fig:setup:overview}
\end{figure}

\partitle{detectors}
The \gerda\ \phasetwo\ array is organized in 7 vertical strings, holding 40 detectors in
total. The detectors can be divided in three groups: the \bege\ detectors, the
semi-coaxial \m{ANG} and \m{RG}, and the semi-coaxial \m{GTF} detectors. The
detectors of the first two groups are made of germanium enriched in \gesix, the third
group includes detectors with natural isotopic germanium abundance.

All \gerda\ HPGe detectors are made of high-purity p-type germanium, which is initially
used to pull crystals, typically fuse-shaped (see for example fig.~4.1a in
\cite{Yonenaga2019}).  Crystals are then cut in slices, and each of them is further
processed to obtain the final detector geometry. The electrodes for signal read-out and
voltage biasing are then fabricated on the detector surface. The \nplus\ contact, where the
external voltage is applied, `wraps around' the detector. It is obtained by deposition of
a lithium layer on the surface, which diffuses below the surface until a depth of
$\sim$1~mm during the subsequent thermal annealing cycles. The presence of lithium
impurities effectively creates a region with decreased charge collection efficiency (CCE),
or `dead-layer', even when biased at full-depletion voltages.  In this region, the CCE is
zero at the surface and reaches its maximal value at the full charge collection depth
(FCCD). The \pplus\ electrode, where the signal is read out, is instead fabricated by boron
implantation, and the dead layer it produces is typically smaller, at the level of
hundreds of microns. The two conductive surfaces are separated by an insulating region,
which is typically produced by excavating a `groove'. In some cases such groove is
passivated by deposition of a germanium-oxide layer.

\partitle{%
  \includegraphics[width=13mm]{gedet/BEGe.png}\\
  \includegraphics[width=13mm]{gedet/SemiCoax.png}%
}
The \gerda\ \phasetwo\ detectors can be classified according to two different geometry
types: semi-coaxial and \bege. In the semi-coaxial design, a bore-hole is excavated along
the central axis to accomodate the \pplus\ electrode. With such a configuration,
relatively large detector masses can be achieved, of the order of 2--3~kg. The \ANG{} (5),
\RG{} (2) and \GTF{} (3) detectors, inherited from the \hdm\ and \igex\ experiments and
already used in \phaseone, are of the semi-coaxial type. Their total mass amounts to
23.2~kg of germanium, while the enrichment fractions are in the 85.5--88.3\% range. For
\phasetwo, 20~kg of germanium enriched at 87.8\% was procured by the \gerda\ collaboration
for the production of 30 new diodes of the \bege\ type. The Broad Energy Germanium detector
design does not include a bore-hole, therefore the \pplus\ contact is a small, dot-shaped
surface at the center of one of the two detector sides. The absence of a bore-hole makes
this kind of detectors harder to fully deplete, requiring lower impurity concentrations
and smaller masses, generally lower than 1~kg. A detailed description of the
characteristics of the \bege\ detectors, from germanium procurement to diode production can
be found in~\cite{Agostini2015e, Agostini2018a, Agostini2019}.

\begin{figure}
  \centering
  \includegraphics[height=7cm]{gedet/phII-array.png}
  \hspace{0.5cm}
  \includegraphics[height=6cm]{gedet/phII-array-2D.pdf}
  \caption{%
    The \gerda\ \phasetwo\ detector array.
  }\label{fig:setup:array}
\end{figure}

\partitle{array \\ instrumentation}
As already mentioned, the \gerda\ \phasetwo\ detectors are arranged into 7 strings, packed
closely together as depicted in \cref{fig:setup:magevolumes}a, to maximize the
multi-detector event rejection efficiency. Since the main background sources in
\phaseone\ were located close to the detectors, the design of the mounting and cabling
system has been carefully chosen to minimize the mass. The detector holder unit consists
of a low-mass, intrinsically radio-pure silicon plate and three vertical copper bars to
take the detector weight and connect the modules between themselves within a string. The
silicon plate provides the substrate onto which signal and high voltage cables are
attached. The Ge detectors are read out with custom-produced, cryogenic and low
radioactivity preamplifiers called `CC3'~\cite{Riboldi2015}. The Ge readout electrode is
connected to the JFET-PCB by a flexible flat cable. Two different cable types are adopted
for the signal and HV contact: the HV cables are made from 10 mils Cuflon\reg, or 3 mils
Pyralux\reg, the signal FFCs from 3 mils Cuflon\reg\ or Pyralux\reg. See
\cref{fig:setup:magevolumes}b.

\partitle{LAr veto}
To improve the sensitivity on the \onbb\ half-life and operate in the \bkgfree\ regime, an
additional active veto system to collect the LAr scintillation light produced by
background events was designed and installed during the upgrade works for \phasetwo. A
cylindrical hybrid design was chosen to detect the light information: a curtain made of
light-guiding plastic fibers coupled to a ring of silicon photomultipliers (SiPMs) to
surround the array and PMTs on the top (9) and on the bottom (7) (see
\cref{fig:setup:magevolumes}f). To enhance the light collection efficiency two copper
shrouds (visible in \cref{fig:setup:magevolumes}e) coated with a reflective Tetratex\reg\
layer were added between the fiber shroud and the PMT holder plates. The latter were
coated with a reflective VM2000 layer. Another light collection improvement introduced by
the \phasetwo\ upgrade is the installation of nylon (mini-)shrouds enclosing each detector
string (\cref{fig:setup:magevolumes}d). The presence of these shrouds provides an
essential mechanical barrier to reduce the background from \kvz\ ions naturally present in
LAr, which undergo \b-decay and can mimic the \onbb\ signature at \qbb. Being made of
transparent nylon material, in contrast to the ones from \phaseone\ made of copper, the
mini-shrouds let the light propagate more efficiently to a close-by light collecting
surface. To match the fibers and PMTs spectral response many surfaces in the close vicinity
of the array were coated with tetraphenil-butadiene (TPB), a wavelength shifting material.
Coating has been applied on mini-shrouds, fiber-shroud, copper shroud, PMTs as well as
their holder plates. The reader is referred to ref.~\cite{Agostini2018a} for the detailed
LAr veto instrumentation technical specifications.

\begin{figure}
  \includegraphics{setup/pic-collage.pdf}
  \caption{%
    Various pictures of the \gerda\ \phasetwo\ setup, taken during the upgrade
    works.
  }\label{fig:setup:pictures}
\end{figure}

\partitle{calibration \\ system}
The \gerda\ weekly calibrations are performed by lowering three \Th\ sources into LAr in
the close vicinity of the array, at the same radial distance from the array central axis
and evenly spaced. Each source, when lowered, just fits into the space between the
cylinder of the LAr veto system and two neighboring outer strings of the detector array,
thereby the sources enter the inner volume of the LAr veto system by three slots in the
top PMT plate.  The three sources were produced for \phasetwo\ and
characterized~\cite{Baudis2015}. The LAr veto instrumentation is usually switched off
during calibration runs because of the too high source activity of $\mathcal{O}(10)$~kBq.
However, less intense \Ra\ sources are also available and can be easily exchanged with the
standard ones. Special calibration data has been acquired with these sources and the LAr
light instrumentation turned on, to study the performance of the LAr veto system. The
calibration of the experimental setup is extensively described in ref.~\fillme{[fillme]}.

\partitle{data \\ acquisition}
A FADC system records traces from germanium detectors (40), PMTs (16) and SiPMs (15) of
the LAr veto, PMTs and scintillating panels of the muon veto when an energy deposition
greater than about 100~keV occurs in at least one of the germanium detectors\footnote{The
  exact trigger threshold is detector- and run-dependent and varies between 20~keV and
200~keV.}.  The energy deposition associated to each germanium detector signal is
determined via a Zero Area Cusp (ZAC) filter which is optimized off-line for each detector
and each calibration run~\cite{Agostini2015}. PMT and SiPM hits are reconstructed in the
offline analysis following the procedure documented in~\cite{Agostini2018a}. Each event
has to pass a series of quality cuts tailored to discard unphysical events. The
reconstructed trigger positions are converted into time differences relative to the first
trigger found in the germanium detector traces. Trigger positions and amplitudes are
subsequently used together with hits from the SiPM to test the LAr veto condition. The
algorithms were implemented in the \gelatio\ framework~\cite{Agostini2011} which is used
to process \gerda\ data. Each event is characterized by the calibrated energy deposited in
the Ge diode, a data quality flag, the classification as signal or background event from
the pulse shape analysis, and veto flags from the muon veto and LAr veto systems.

\begin{figure}
  \centering
  \includegraphics[width=\textwidth]{setup/mage-volumes.pdf}
  \caption{%
    Implementation of the \gerda\ array in \mage, visualized using the
    \geant\ visualization drivers. From left to right: a) the \gerda\
    detectors, b) the holder mounting, composed of silicon plates and
    copper bars c) the high-voltage and signal flexible flat cables plus
    the front-end electronics on top, d) the full array instrumentation,
    including the transparent nylon mini-shrouds, e) the full LAr veto
    system surrounding the array, including the fiber shroud (in green),
    the Tetratex\reg-coated copper shrouds (above and below the fibers) and
    the two PMT arrays, f) the LAr veto system without the copper
    shrouds.%
  }\label{fig:setup:magevolumes}
\end{figure}

\section{Background reduction techniques}%
\label{sec:gerda:cuts}

\begin{figure}
  \centering
  \includegraphics[width=\textwidth]{gedet/gerda-events.png}
  \caption{%
    Event types in \gerda.
  }\label{fig:gerda:event-types}
\end{figure}

Various background mitigation techniques are adopted, both at the data acquisition level
(online) and the analysis level (offline) in \gerda\ to lower the background index to the
`background-free' level of \pIIbi.

\partitle{muon veto}
Muons may cause a substantial background to rare event searches like \gerda\ by generating
counts at \qbb\ either through direct energy deposition in the detectors or through
e.g.~decay radiation of spallation products. At LNGS the cosmic muon flux is reduced by a
factor of ${\sim}10^6$ to a rate of ${\sim}3.4 \cdot 10^{−4}~\text{s}^{-1}\text{m}^{-2}$,
which is sill sufficient to generate a non-negligible background of the order of
\powctsper{-3}.  As already described in \cref{sec:gerda:setup}, a muon veto comprising of
a water \v{C}erenkov veto and a scintillator veto was implemented in \gerda\ to reduce
this background contribution. An event with energy deposition in germanium is flagged
as muon-induced background if a coincidence with the muon veto signal occurs in a $\pm
10$~\mus\ window around the germanium trigger. The efficiency of the muon veto system
has been estimated to be of ${\sim}99$\%, leading to a residual background index of
${\sim}$\powctsper{-5}~\cite{Freund2016}.

\partitle{LAr veto}
The primary role of liquid argon in \gerda\ is to keep the germanium detectors at a
cryogenic operational temperature and provide a passive shielding medium against external
backgrounds. Moreover, the LAr can be also employed as a detector medium in an active veto
system, thanks to its scintillation properties. The production mechanism of the
scintillation light in LAr is known since several decades and is deeply described in
literature and its spectrum is today well known. The incident particles deposit their
energy mainly by interactions with the electron shell of the argon atoms which leads to
either an excitation or an ionization of argon atoms. Excited argon atoms are frequently
called `excited dimers' or `excimers' in the literature. Their decay is accompanied by the
emission of scintillation light in the vacuum ultraviolet region, whose typical wavelength
is usually cited as $\lambda = 128$~nm~\cite{Heindl2010}. The ratio between excitation and
ionization is strongly dependent on the pressure and density of the argon as well as on
the type of radiation itself. In the case of excitation, the excited argon atom can
directly form an excimer via the collision with neighboring argon atoms. The process is
sketched in \cref{fig:setup:lar-scint}.
\begin{figure}[h]
  \centering
  \includegraphics[width=0.7\linewidth]{lar-scint-mechanism.pdf}
  \caption{%
    Scintillation mechanism of liquid argon (or gaseous argon) via the decay of excited
    dimers. The excited dimer can be either formed directly from an excited argon atom or
    from an ionized atom which forms an ionized dimer before its recombination and the
    following recombination in its molecular form. Drawing courtesy of Christoph
    Wiesinger.
  }\label{fig:setup:lar-scint}
\end{figure}
The excimer itself is meta-stable and appears in two different states: the singlet and the
triplet state~\cite{Jortner1965, McCusker1984}.  The decay of the triplet state is
forbidden due to angular momentum conservation, while the decay of the singlet state is
allowed. Consequently the lifetime of the triplet state is 1.59~\mus\ which is
significantly higher than the 6~ns of the singlet state. The scintillation light yield
(combined for both components) is roughly 40 photons/keV, measured in ultra-pure
LAr~\cite{Doke1988}. This value is dependent on different factors, like the presence of
contaminants, the pressure and density of the argon as well as the ionization density of
the incident particle~\cite{Doke1988}.

The goal of the \gerda\ LAr veto is to reject those types of background events in the
germanium detectors that simultaneously deposit energy in the surrounding LAr, and hence
generate scintillation. These background types mainly include \g-ray background from Ra
and Th decays in solid materials inside and around the detectors. But also other types of
background can successfully be rejected, such as muons or decays from \Arh\ or \kvz. An event
depositing energy in the germanium detectors is discarded as background if a coincidence
with the LAr veto signal is found in the time window spanned by the germanium traces.
Since the lifetime of the LAr triplet state significantly depends on the argon
purity~\cite{Amsler2007}, it is possible to monitor the purity of LAr over time.
\cref{fig:lar:triplet-lifetime} shows the lifetime values measured every month since
the start of \phasetwo. \fillme{comment}.
\begin{figure}
  \centering
  \includegraphics[width=0.8\linewidth]{plots/lar-triplet-lifetime.png}
  \caption{%
    \fillme{to be updated} LAr triplet lifetime regularly measured during \phasetwo.
  }\label{fig:lar:triplet-lifetime}
\end{figure}
The efficiency of the LAr veto cut can be estimated by evaluating the number of test
pulses that are randomly flagged as background events. For \phasetwo\ this fraction has
been evaluated to 97\%, corresponding to an efficiency for \onbb\ events of
\fillme{fillme}.

% vim: tw=90
