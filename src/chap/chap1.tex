\chapter{Double-beta decay}\label{chap:theory}
\pagenumbering{arabic}

In this section we briefly review the theory of double-beta decay in its two
most studied modes, the two-neutrino and the neutrino less one, together with
the mode which is relevant to this work, the Lorentz-violating one. We show the
formulas for the differential decay rates associated to the two-neutrino and
the Lorentz-violating modes that will be cited in the following sections.
Finally the current experimental knowledge about the Lorentz-violating mode is
presented.

\section{The two-neutrino double-beta decay}
The two-neutrino double-beta decay (\nnbb) processes, first suggested by
M.~Goeppert-Mayer in 1935 \cite{GoeppertMayer1935}, can be schematically defined
as:
\[
  \begin{array}{lrl}
    \mathcal{N}(A,Z)\longrightarrow \mathcal{N}(A,Z+2)+2e^-+2\bar{\nu}_e &
      \qquad [2\nu\beta^-\beta^-] & \\
    \mathcal{N}(A,Z)\longrightarrow \mathcal{N}(A,Z-2)+2e^++2\nu_e &
      \qquad [2\nu\beta^+\beta^+] & ,\\
  \end{array}
\]
where $\mathcal{N}(A,Z)$ represents a nucleus with mass number $A$ and atomic
number $Z$. A \nnbbm\ (\nnbbp) process consists of
the simultaneous $\beta^-$ ($\beta^+$) decay of two neutrons (protons) in the
same nucleus. The processes are generated at second-order in the perturbative
expansion of weak interactions in the Standard Model. The Feynman graph for
\nnbbm\ is shown in Fig.~\cref{fig:nbb:feydiag}, left.

Since the \nnbb\ decays have a four-body leptonic final state, the sum of the
kinetic energies of the two decay electrons have a continuous spectrum from
zero to the Q-value of the decay process (the recoil energy of the final
nucleus is negligible), which is given by
\begin{equation}
  Q_{\beta\beta} = M_i - M_f - 2m_e \;,
\end{equation}
where $M_i$ and $M_f$ are, respectively, the masses of the initial and final
nuclei (i.e. the energy levels of their ground states; if the transition occurs
into an excited energy level of the final nucleus, $M_f$ must be replaced with
the appropriate energy).

A nucleus $\mathcal{N}(A,Z)$ can decay through a \nnbb\ process if its ground
state has an energy which is larger than the ground-state energy of the nucleus
$\mathcal{N}(A,Z\pm2)$ plus twice the electron mass. Moreover, if a nucleus can
decay through both the $\beta$ and \nnbb processes, in practice the latter is
not observable, because its $\beta$ decay lifetime is much shorter than its
\nnbb\ decay lifetime (the half-life of \nnbb\ is typically around
$10^{19}-10^{24}$ yrs). Therefore, in practice the \nnbb\ decay of a nucleus
is observable only if its $\beta$ decay is energetically forbidden or strongly
suppressed because of a large change of spin. The $\beta^-$ decay of a nucleus
$\mathcal{N}(A,Z)$ is energetically forbidden if its ground-state energy is
lower than the ground-state energy of the nucleus $\mathcal{N}(A,Z+1)$ plus the
electron mass ($Q_{\beta^{-}}<0$). Typically, in \nnbbm\ decays
the energy levels of the three nuclei $\mathcal{N}(A,Z)$, $\mathcal{N}(A,Z+1)$,
and $\mathcal{N}(A,Z+2)$ are of the type depicted in
Fig.~\cref{fig:nbb:gesixlevels}, left, where the specific case of \gesix,
$^{76}$As, and $^{76}$Se nuclei is considered.

\begin{figure}
  \centering%
  \makebox[\textwidth]{%
    \includegraphics[width=0.5\textwidth]{plots/2nbbfey.pdf}%
    \includegraphics[width=0.5\textwidth]{plots/0nbbfey.pdf}%
  }
  \caption{%
    Feynman graphs for two-neutrino (left) and neutrinoless (right) double-beta
    decay.
  }\label{fig:nbb:feydiag}
\end{figure}

\begin{figure}
  \centering
  \makebox[\textwidth]{%
    \includegraphics[width=0.5\textwidth]{plots/gesix-levels.pdf}%
    \includegraphics[width=0.5\textwidth]{plots/masspar.pdf}%
  }%
  \caption{%
    On the left: schematic illustration of the energy level structure of the
    $2\nu\beta^-\beta^-$ decay of \gesix into $^{76}$Se. On the right: general
    energy level configuration for double-beta decay emitters.  The situation
    for a nucleus with even mass number $A$ is presented: the mass parabola,
    representing the dependence of the binding energy $M(A,Z)$ on the atomic
    number $Z$, is plotted for even-even (even number of protons and neutrons)
    and odd-odd nuclei with the relevant \b\ and \b\b\ decays among them.
  }\label{fig:nbb:gesixlevels}
\end{figure}
