%!TEX root = ../main.tex

\chapter{\texorpdfstring{Time distribution of \a\ events}{Time distribution of alpha-events}}%
\label{apdx:timealpha}

The time distribution of \Po\ decays is well known to be exponential. In the presence of a
\Pbl\ contamination, however, an additional constant contribution can also be observed.
\Pbl, decaying to \Po, feeds a constant \Po\ component once their decay rates stabilize in
a secular equilibrium. To disentangle the two, a fit of the time distribution of events
with energies between 3.5~MeV and 5.3~MeV observed in the \gerda\ data is performed, with
a constant $C$ and an exponential function:
\[
  f(t) = C + N \exp\left( - \frac{\log2}{T_{1/2}}t \right) \;,
\]
where $T_{1/2}=(138.4\pm0.2)$~days is the half-life of \Po. A Poisson likelihood function
corrected for data acquisition dead time~\cite{Cleveland1983} is used, in which the bin
contents are modeled as follows:
\[
  \nu_i = f_i^{\mathrm{LT}}
  \left\{ C \delta t + N \tau
    \left[
      \exp\left( -\frac{t_0 + i \delta t}{\tau} \right)  -
      \exp\left( -\frac{t_0 + (i+1) \delta t}{\tau} \right)
    \right]
  \right\} \;,
\]
where $C$ and $N$ are the amplitudes of the constant and the exponentially-decaying
components respectively and are the only two free fit parameters.  $f_i^{\mathrm{LT}}$ is
the live-time fraction in time-bin $i$ which is estimated from injected test pulser
events, $\delta t$ is the time-bin width and $\tau = T_{1/2} / \log2$.
\newpar
The log-likelihood can be expressed as a sum:
\[
  \log \mathcal{L}_\alpha^\text{time}(C,N \,|\, n) =
  \sum_{i=1}^{N_\text{bins}} n_i \cdot
  \log\nu_i - \nu_i - \log n_i! \;.
\]
Only detectors that were ON or in anti-coincidence mode\footnote{%
  Detectors in anti-coincidence mode are not well energy-calibrated and generally
  discarded in data analysis. Here, we are not interested in the precise energy of an
  event because the selected energy window is large with respect to a possible
  mis-calibration.
} in the full data taking period have been selected. 27 \bege{}s, 5 \scoax{}s and 4
\icoax{}s result eligible. In this way, bias due to selection or de-selection of
particularly contaminated detectors is avoided. Furthermore, the initial data-taking
period between December \yr{2015} to January \yr{2016} is excluded from the present analysis
because of detector instabilities after the \phasetwo\ upgrade works. The analyzed data
spans the period between the end of January \yr{2016} to the end of November \yr{2019} and
is split into five data sets according to detector type (\bege, \scoax\ and \icoax) and
experimental phase (\phaseone\ and \phasetwop).
\newpar
The fit results are shown in \autoref{fig:bkg:raw:timealpha:results}. For the \bege\ data
set, about half of the initial contamination decays exponentially, while for the \scoax\
data set the ratio of $N$ to $C$ is about 5 to 1. After several \Po\ half-lives a stable
rate of $\sim1~\upalpha$/day is expected in either data set. A comparison between the fit
results before and after the \phasetwop\ upgrade demonstrates the achievement in detector
handling techniques during the upgrade works. As a matter of fact, the constant fit
component, representing the long-lived \Po\ contamination, is not affected by the manual
intervention in \yr{2018}. Moreover, the initial level of \Pbl\ is lower at the beginning
of \phasetwop, compared to the beginning of \phasetwo. The \a-emitter contamination at the
\icoax\ surface, as one may notice also in \cref{fig:bkg:raw:ph2p:data-desc}, is observed
to be particularly low.

\begin{figure}
  \centering
  \includegraphics{plots/bkg/raw/ph2/results/amodel/amodel-timealpha.pdf}
  \caption{%
    \a\ events time distribution in the $[3.5,5.4]$~MeV energy range during the whole
    \phasetwo\ data taking. Each data point correspond to events cumulated in 20~days of
    effective run time. Five data sets are considered: 27 \bege\ detectors in \phasetwo\ and
    \phasetwop\ (top panel), 5 \scoax\ detectors in \phasetwo\ and \phasetwop, and 4
    \icoax\ detectors in \phasetwop\ (bottom panel). An exponential model is fit to each
    data set and the results are overlaid on the plot.
  }\label{fig:bkg:raw:timealpha:results}
\end{figure}

% vim: tw=90
