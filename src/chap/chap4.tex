%!TEX root = ../main.tex

\chapter{The background after the LAr veto cut}\label{chap:bkg:lar:ph2}

All what has been shown until now concerns data before the LAr and PSD cut. In this
chapter a model of the background after the LAr veto cut will be presented, based on a
Monte Carlo simulation of the LAr scintillation light propagation. Being able to describe
the background after this major event selection is indeed of great interest to study the
distribution of two-neutrino double-beta decay events, which are almost never vetoed by
the LAr veto system. As extensively shown in \cref{chap:theory}, the presence of several
new physics phenomena can be constrained by looking at the shape of the \nnbb\ events
distribution. Understanding the action of the LAr veto cut on background events from the
point of view of the background model requires, however, a full Monte Carlo simulation of
the LAr scintillation mechanism as well as the implementation of all the relevant material
and surface optical properties that contribute to light propagation in the \gerda\ experimental
setup. Implementing such a simulation, as it will be shown, requires an accurate knowledge
of many optical parameters, which is not always the case, unfortunately, with the \gerda\
setup. Nevertheless, it is possible to use special calibration data with low-activity
sources and the LAr veto instrumentation turned on to constrain the LAr veto Monte Carlo
model. An independent analysis of this special data set is used to tune the unknown
optical parameters in the Monte Carlo such to reproduce the observed vetoing performance.
The obtained parameters are then used to produce a map of the LAr scintillation light
detection probability, which is applied to the background model simulations in order to
obtain the LAr veto flag. Based on these new background model pdfs, a statistical analysis
to test possible deviations of the \nnbb\ distribution from its Standard Model description
will be finally presented.
\newpar
The chapter is structured as follows: \fillme{fillme}

\section{Optical physics in \mage}%
\label{sec:bkg:lar:ph2:mage}

Optical materials and surfaces are implemented in \mage\ through the relevant \geant\
libraries. Once properties like reflectivity, wavelength-shifting capabilities,
attenuation lengths, scintillation yields etc.~are defined, the \geant\ core routines take
care of simulating light propagation accordingly. In the following a reference list of the
optical properties implemented in \mage\ for this study, with references to the
literature.

\subsection{Liquid argon}
Key properties of the liquid argon from the point of view of the vetoing performance in
\gerda\ are the scintillation mechanism, the refractive index and the attenuation length.
The first two are relatively well known, while the latter strongly depends on the LAr
purity, which has not been measured for \gerda. We recall here that the deployed LAr has
not been subject to any purification process, and is therefore expected to meet the purity
specifications of natural argon.

\begin{description}

  \item[Refractive index] shown in \cref{fig:bkg:lar:ph2:mage:lar-props}, formulas are
    taken from~\cite{Bideau-Mehu1981}.

  \item[Rayleigh scattering length] shown in \cref{fig:bkg:lar:ph2:mage:lar-props},
    calculations are taken from~\cite{Seidel2002}. The refractive index calculation
    mentioned before is used.

  \item[Scintillation spectrum] in~\cite{Heindl2010} a measured scintillation spectrum
    is shown (fig.~1 and 2). Only the (gaussian) peak is implemented in \mage\
    (\cref{fig:bkg:lar:ph2:mage:lar-props}), as the non-gaussian contributions are of several
    orders of magnitude lower and are therefore considered negligible.  A normal
    distribution $(\mu=128\;\text{nm}, \sigma=2.929\;\text{nm})$ is implemented.

  \item[Scintillation yield] different scintillation processes are defined in the \mage\
    physics list for different ionizing particles, that in general show different
    scintillation yields in LAr. A default, reference yield value of
    $51\;\upgamma/\text{keV}$ is set for all particles~\cite{Doke2002}. This value does
    certainly not represent the reality of \gerda, as the yield strongly depends on the
    electric field configuration and the quencher impurities. Unfortunately a a reliable
    direct measure of the scintillation yield of the \gerda\ LAr is not available.
    Indirect measurements were performed with \large~\cite{Lehnert2015} and with a
    dedicated setup deployed inside the \gerda\ cryostat~\cite{Barros2020} yielding
    incompatible results. As shown in~\cite{Doke2002}, some particles, interestingly \a\
    particles and nuclei, can have lower yields.  Therefore, the photon yield is reduced
    for \a-particles and nuclei by a factor of 0.875 and 0.375 respectively in the physics
    list. These numbers are extracted from~\cite{Doke2002}. In this way \b\ and \g\
    particles will be affected by the nominal (maximum) photon yield, while \a\ particles
    and nuclei will produce less light by a factor 0.875 and 0.375 (0.7/0.8 and 0.3/0.8
    according to~\cite{Doke2002}) respectively

  \item[Singlet and triplet lifetime] \sloppy The implemented triplet lifetime is the one
    measured during \gerdatwo\ (\cref{fig:lar:triplet-lifetime}), the singlet lifetime
    \fillme{fillme}. The relative occurrence of the two de-excitation processes is also
    specified in terms of scintillation yield. The \geant\ \m{YIELDRATIO} property, which
    is defined as the relative strength of the fast component as a fraction of total
    scintillation yield, is set to 0.23 for all particles (\b\ and \g\ particles), 0.75
    for nuclei and 1.0 for \a\ particles (the latter is a rough guess).

  \item[Attenuation length] Strongly dependent on the LAr purity, therefore no literature
    values can be used. It depends on the wavelength of the photon in general, but this
    dependence is poorly known in the VUV regime.  The most important for \gerda\ is its
    value at 128~nm, the wavelength of the LAr scintillation light. The following
    implementation is adopted: an exponential function is used to make sure that the LAr
    is opaque to photons with higher energy than 128~nm but transparent to WLS photons
    (lower energy), see fig.~\ref{fig:bkg:lar:ph2:mage:lar-props}. A measurement of the
    attenuation length in the \gerda\ LAr was performed~\cite{Barros2020}, yielding
    $\sim15$~cm as a result, but \fillme{fillme}

  \item[Fano factor] a Fano factor of 0.11 is set, taken from~\cite{Doke1976}.

\end{description}

\begin{figure}
  \centering
  \includegraphics{plots/mage/lar-props.pdf}
  \caption{%
    LAr optical properties as implemented in \mage. Top left: the refractive index, top
    right: the Rayleigh absorption length, bottom left: the scintillation spectrum, bottom
    right: the absorption length. Taken from~\cite{Bideau-Mehu1981, Seidel2002,
    Heindl2010}.
  }\label{fig:bkg:lar:ph2:mage:lar-props}
\end{figure}

\subsection{Reflectivity}

Measurements are taken from Anne Wegmann's Ph.D.~thesis~\cite{Wegmann2017}. A
reflectometer at room temperature and in air was used to measure the reflectivity in the
wavelength range $[280, 700]$~nm. Values in the VUV region must be taken from other
sources. For germanium it seems to strongly depend upon the radiation incident
angle~\cite{Marton1967}, but it's not possible to implement angle-dependent reflectivities
in \geant\ yet.  A rough, average value of 0.65 is therefore set for wavelengths smaller
than 280~nm.  The source of the reflectivity values of the other materials below 280~nm is
not known to me \fillme{fillme}.  The reflectivity values for all materials mentioned
above are plot in \cref{fig:bkg:lar:ph2:mage:metals-refl}

\begin{figure}
  \centering
  \includegraphics{plots/mage/metals-refl.pdf}
  \caption{%
    Reflectivity of germanium, copper, silicon and Teflon as implemented in \mage.
  }\label{fig:bkg:lar:ph2:mage:metals-refl}
\end{figure}

Tetratex\reg{} values are taken from~\cite{Janecek2012}. The author reports measurements
of the reflectivity of 2 and 4 superimposed layers of 160~\mum\ thick Tetratex. As the
thickness of the foils used in \gerda\ is 254~\mum, the results for the two superimposed
foils (320~\mum) are implemented in \mage. In reality, the reflectivity of the \gerda\
foils should be (negligibly) smaller. The TPB layer has some effect on the reflectivity,
but there's no measurement available in literature.  VM2000 values are taken
from~\cite{Francini2013}.  The authors report measurements of TPB-coated VM2000, like in
the \gerda\ setup. The measurement seems to be done properly as they take into account the
effect of the TPB emission spectrum.  The measurement seems to be independent on the TPB
layer thickness. The values are plot in \cref{fig:bkg:lar:ph2:mage:misc}.

\begin{figure}
  \centering
  \includegraphics{plots/mage/misc.pdf}
  \caption{%
    as implemented in \mage.
  }\label{fig:bkg:lar:ph2:mage:misc}
\end{figure}

% vim: tw=90
