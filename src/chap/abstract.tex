%!TEX root = main.tex

The search for neutrinoless double-beta decay is generally quoted as the only
practical way to establish the nature of the neutrino mass, one of the most
elusive and intriguing elementary particles in the Standard Model. The detection
of this rare nuclear decay would attribute to neutrino special properties which
have been described for the first time by Ettore Majorana at the beginning of
the last century. A discovery would point up to the inadequacy of current
fundamental physics theories. According to some other theories, it might even
contribute to solve the mystery of the asymmetry between matter and anti-matter
in our universe. For more than fifty years, neutrinoless double-beta decay has
been searched for, unsuccessfully, in the isotope of germanium with 76 nucleons.
The \gerda\ experiment, officially closed in November \yr{2019}, has been a
pioneer in the field, having demonstrated the maturity of its experimental
technology to realize a tonne-scale experiment, capable of setting limits on the
order of \powtenyr{27} on the decay half-life. This thesis project aims to prove
the \gerda\ excellent results in terms of background event level, developing a
statistical model of the latter. In particular, a Monte Carlo simulation of the
liquid argon veto cut, based on the scintillation light emitted in presence of
ionizing particles, will be presented for the first time. The results of this
model will be employed in the study of the two-neutrino double-beta decay energy
distribution, to measure the process half-life and to search for new-physics
signals, originating from a hypothetical Majoron emission or from the presence
of Lorentz symmetry and CPT violations. The estimated two-neutrino double-beta
decay half-life is $(0.0 \pm \stat{0.0} \pm \syst{0.0}) \cdot 10^{21}$~yr, while
a lower limit at 90\% C.L.~on the neutrinoless double-beta decay with Majoron
emission (spectral index $n=1$) is set at $0 \cdot 10^{23}$~yr. These results
substantially improve previous estimates with \gesix\ in terms of precision and
sensitivity.

% vim: tw=90
