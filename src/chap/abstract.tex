%!TEX root = main.tex

The search for neutrinoless double-beta decay is generally quoted as the only practical
way to establish the nature of the mass of neutrino, one of the most elusive and intriguing
elementary particles in the Standard Model. The detection of this rare nuclear decay would
attribute to neutrino special properties, described for the first time by Ettore Majorana
at the beginning of the last century. A discovery would decisively prove the inadequacy of
current fundamental physics theories, in favour of more general formulations. According to
some of these novel theories, it might even contribute to solve the mystery of the
asymmetry between matter and anti-matter in our universe. For more than fifty years,
neutrinoless double-beta decay has been unsuccessfully searched for in the isotope of
germanium with 76 nucleons.  The \gerda\ experiment, officially concluded in November
\yr{2019}, has been a pioneer of the field. It demonstrated the maturity of the germanium
experimental technology to realize a tonne-scale experiment, capable of setting limits on
the order of \powtenyr{27} on the decay half-life. This thesis project aims to attest the
excellent \gerda\ results in terms of background event level, developing a statistical
model of the latter. In particular, a Monte Carlo simulation of the liquid argon veto cut,
based on the scintillation light emitted at the passage of ionizing particles, will be
presented for the first time. The results of this model will be employed in the study of
the two-neutrino double-beta decay energy distribution, to measure the process half-life
and to search for new-physics signals, originating from a hypothetical Majoron emission or
from the presence of Lorentz symmetry and CPT violations. The estimated two-neutrino
double-beta decay half-life is $(\fillme{?} \pm \stat{\fillme{?}} \pm \syst{\fillme{?}})
\cdot 10^{21}$~yr, while a lower limit at 90\% C.L.~on the neutrinoless double-beta decay
with Majoron emission (spectral index $n=1$) is set at $\fillme{?} \cdot 10^{23}$~yr.
These results substantially improve previous estimates with \gesix\ in terms of precision
and sensitivity.

% vim: tw=90
