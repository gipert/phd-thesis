%!TEX root = ../main.tex

\chapter{Assay measurements}%
\label{apdx:assay}

To assess whether construction materials meet the \gerda\ radio-purity requirements,
screening measurement campaigns have been carried out before the \phasetwo\ and \phasetwop\
upgrades. The extracted radioactive decay activities constitute an important set of prior
information that guides the construction of the background model presented in
\cref{chap:bkg:raw:ph2} and following chapters. The results of the measurements are
converted to useful units for the characterisation of prior distributions in the
background model and other analyses. Part of these measurements are published also
in~\cite{Agostini2018a}.
\newpar
Radio-purity was measured with three techniques: \g\ spectroscopy, ICP-MS and neutron
activation analysis (NAA), which main characteristics are summarized below.
\begin{description}
  \item[\g\ spectroscopy] is performed with photon-counting apparata (usually germanium
    detectors). The radio-activity of a material sample is determined by measuring the
    intensity of known \g\ lines from various decaying isotopes. \g\ spectroscopy is
    well suited to constrain \kvn\ contaminations.
  \item[ICP-MS] (Inductively Coupled Plasma Mass Spectrometry) is a technique that ionizes
    the material samples and counts the resulting ions after separating them according to
    their charge-to-mass ratio. It is applicable to materials which can be easily put in
    liquid form. Statistical uncertainties on ICP-MS measurements are usually small, but
    systematic effects can bias the results.
  \item[NAA] (Neutron Activation Analysis) essentially exposes the target material sample
    to a flux of neutrons. The \g\ radiation emitted by the activated nuclei in the sample
    is analyzed to determine the initial isotopic composition. Light atomic nuclei
    (e.g.~potassium) which are not activated by neutrons cannot be detected by this
    method.
\end{description}

The results of all screening measurements available for \gerdatwo\ materials are
summarized in \cref{tab:bkg:ph2:assay,tab:bkg:ph2p:assay}. Recommendations on how to
properly use the values as prior information in the background model are included. Note
that the following groups of isotopes are assumed to be in secular equilibrium:
\[
  ^{238}\text{U}  \succ ^{234\text{m}}\text{Pa}               \;\; \| \;\;
  ^{226}\text{Ra} \succ ^{214}\text{Pb} \succ ^{214}\text{Bi} \;\; \| \;\;
  ^{228}\text{Ra} \succ ^{228}\text{Ac}                       \;\; \| \;\;
  ^{228}\text{Th} \succ ^{212}\text{Bi} \succ ^{208}\text{Tl} \;.
\]

\clearpage
\thispagestyle{empty}
\begin{sidewaystable}
  \caption{%
    Activity of \gerda\ \phasetwo\ setup components calculated from radio-purity screening
    measurements. Values in gray can be used as priors for \phasetwo\ background modeling
    as they are; values in green are (partly) derived from ICP-MS/NAA measurements under
    the assumption of secular equilibrium in the U/Th decay chains; values in light blue
    have to be multiplied by 2/3. The latter is an approximation for the shortening of the
    cables before deployment in \gerda.
  }\label{tab:bkg:ph2:assay}
  \definecolor{TolGray}      {HTML}{BBBBBB}
\definecolor{TolPaleGray}  {HTML}{DDDDDD}

\definecolor{TolBriBlue}   {HTML}{4477AA}
\definecolor{TolBriCyan}   {HTML}{66CCEE}
\definecolor{TolBriGreen}  {HTML}{228833}
\definecolor{TolBriYellow} {HTML}{CCBB44}
\definecolor{TolBriRed}    {HTML}{EE6677}
\definecolor{TolBriPurple} {HTML}{AA3377}

\definecolor{TolHCYellow}  {HTML}{DDAA33}
\definecolor{TolHCRed}     {HTML}{BB5566}
\definecolor{TolHCBlue}    {HTML}{004488}

\definecolor{TolVibBlue}   {HTML}{0077BB}
\definecolor{TolVibCyan}   {HTML}{33BBEE}
\definecolor{TolVibTeal}   {HTML}{009988}
\definecolor{TolVibOrange} {HTML}{EE7733}
\definecolor{TolVibRed}    {HTML}{CC3311}
\definecolor{TolVibMagenta}{HTML}{EE3377}

\definecolor{TolMutIndigo} {HTML}{332288}
\definecolor{TolMutCyan}   {HTML}{88CCEE}
\definecolor{TolMutTeal}   {HTML}{44AA99}
\definecolor{TolMutGreen}  {HTML}{117733}
\definecolor{TolMutOlive}  {HTML}{999933}
\definecolor{TolMutSand}   {HTML}{DDCC77}
\definecolor{TolMutRose}   {HTML}{CC6677}
\definecolor{TolMutWine}   {HTML}{882255}
\definecolor{TolMutPurple} {HTML}{AA4499}

\definecolor{TolLigBlue}   {HTML}{77AADD}
\definecolor{TolLigCyan}   {HTML}{99DDFF}
\definecolor{TolLigMint}   {HTML}{44BB99}
\definecolor{TolLigPear}   {HTML}{BBCC33}
\definecolor{TolLigOlive}  {HTML}{AAAA00}
\definecolor{TolLigYellow} {HTML}{EEDD88}
\definecolor{TolLigOrange} {HTML}{EE8866}
\definecolor{TolLigPink}   {HTML}{FFAABB}

\newcommand{\rot}[1]{\begin{rotate}{50}#1\end{rotate}}
\newcommand{\rcsg}{\rowcolor{TolLigMint}}
\newcommand{\rclb}{\rowcolor{TolLigBlue}}
\newcommand{\ccsg}{\cellcolor{TolPaleGray}}
\newcommand{\ccw}{\cellcolor{white}}

\begin{tabular}{rllcccccc}
  \rowcolor{white}
  Part        & Material               & Method                    & {$^{228}$Ra (\mubq)}    & {\Ra\ (\mubq)}          & {\Th\ (\mubq)}           & {\Co\ (\mubq)}     & {\kvn\ (mBq)}     & {\Uh\ (mBq)}         \\
  \midrule
  \ccw holders& \ccw Si V IKZ          & \ccw $\gamma$ spec.       & $<250$                  & \ccw $<130$             & \ccw $<96$               & $<100$             & $2.75\pm0.58$     & $<6.2$               \\
              & Si                     & NAA                       & -                       & \ccsg$<6.4\cdot10^{-5}$ & \ccsg $<6.4\cdot10^{-4}$ & -                  & -                 & -                    \\
  cables      & Haefele 10mil          & $\gamma$ spec.            & $<230$                  & $210\pm60$              & $<150$                   & $60\pm30$          & $3.00\pm0.60$     & $<10$                \\
              & Haefele 2mil           & $\gamma$ spec.            & $<320$                  & $780\pm170$             & $<380$                   & $<270$             & $3.5\pm1.7$       & $<35$                \\
              & Tecnomec 3mil          & $\gamma$ spec.            & $<73$                   & $<49$                   & $<060$                   & $<10$              & $1.43\pm0.39$     & $<5.7$               \\
              & Tecnomec 2mil          &                           & $<42$                   & $<29$                   & $<35$                    & $<6$               & $0.83\pm0.23$     & $<3.3$               \\
  \rclb \ccw  & 4 above  \ccw          & \ccw                      & $<660$                  & $990\pm310$             & $<620$                   & $60\pm310$         & $8.7\pm3.0$       & $<54$                \\
  front end   & \ccw CC3               & \ccw $\gamma$ spec.       & $760\pm380$             & $2670\pm380$            & $<1300$                  & $<300$             & $13.3\pm3.8$      & $<21$                \\
  mini-       & coated                 & ICPMS                     & $10.8\pm3.2$            & -                       & $10.8\pm3.2$             & -                  & \ccsg$<0.11$      & $0.0201\pm0.0060$    \\
  shroud      & glued                  & ICPMS                     & $7.2\pm2.2$             & -                       & $7.2\pm2.2$              & -                  & \ccsg$>1.7$       & $0.0214\pm0.0064$    \\
              & 2 above                &                           & \ccsg$18\pm5$           & -                       & \ccsg$18\pm5$            & -                  & $>1.7$            & \ccsg$0.043\pm0.013$ \\
  LAr veto    & fibers                 & ICPMS                     & $44\pm4$                & -                       & $44\pm4$                 & -                  & $0.350\pm0.070$   & $0.032\pm0.003$      \\
              & copper supports        &                           & $130\pm130$             & $520\pm520$             & $130\pm130$              & -                  & -                 & -                    \\
  \rcsg \ccw  & \ccw2 above            & \ccw                      & $180\pm140$             & $520\pm520$             & $180\pm140$              & \ccw -             & $0.350\pm0.070$   & $0.032\pm0.003$      \\
              & SiPM                   & ICPMS                     & $<1.3$                  & $<3.9$                  & $<1.3$                   & -                  & -                 & -                    \\
              & plastic opt. coupling  & $\gamma$ spec.            & -                       & $<6.1$                  & $4.8\pm0.08$             & -                  & $0.0960\pm0.0096$ & -                    \\
              & Cuflon                 & $\gamma$ spec.            & -                       & $19.5\pm6$              & $12\pm0.5$               & -                  & $0.270\pm0.027$   & -                    \\
              & pins                   & $\gamma$ spec.            & -                       & $141\pm28$              & $<55$                    & -                  & $2.07\pm0.20$     & -                    \\
              & screws                 & $\gamma$ spec.            & -                       & $191\pm51$              & $<74$                    & -                  & $<1.5$            & -                    \\
              & glue EP601             & $\gamma$ spec.            & $<6$                    & $<2.2$                  & $<6$                     & -                  & $<0.052$          & -                    \\
  \rcsg \ccw  & \ccw 6 above           &  \ccw                     & $<7.3$                  & $351\pm97$              & $16.8\pm140$             & \ccw -             & $2.4\pm1.8$       & \ccw -               \\
              & PMT top                & $\gamma$ spec.            & -                       & $<15\cdot10^3$          & $<18\cdot10^3$           & -                  & $<82$             & -                    \\
              & PMT bottom             & $\gamma$ spec.            & -                       & $<12\cdot10^3$          & $<14\cdot10^3$           & -                  & $<64$             & -                    \\
              & voltage dividers (VD)  & $\gamma$ spec.            & -                       & $<18\cdot10^3$          & $<8\cdot10^3$            & -                  & $<180$            & -                    \\
  \rcsg \ccw  & \ccw PMT top + 9xVD    &  \ccw                     & \ccw -                  & $<26\cdot10^3$          & $<22\cdot10^3$           & \ccw -             & $<190$            & \ccw -               \\
  \rcsg \ccw  & \ccw PMT bottom + 7xVD &  \ccw                     & \ccw -                  & $<20\cdot10^3$          & $<17\cdot10^3$           & \ccw -             & $<140$            & \ccw -               \\
              & copper shrouds         & ICPMS                     & $62\pm6$                & $250\pm25$              & $62\pm6$                 & -                  & -                 & -                    \\
              & tetratex               & $\gamma$ spec.            & -                       & $282\pm28$              & $132\pm13$               & -                  & $18.4\pm1.8$      & -                    \\
  \rcsg \ccw  & \ccw 2 above           & \ccw                      & $62\pm6$                & $532\pm53$              & $194\pm19$               & \ccw -             & $18.4\pm1.8$      & \ccw -               \\
  \bottomrule
\end{tabular}

\end{sidewaystable}

\begin{sidewaystable}
  \caption{%
    Activity of additional setup components deployed during the \gerda\ \phasetwop\
    upgrade calculated from radio-purity screening measurements. Values in gray can be
    used as priors for \phasetwo\ background modeling as they are; values in green are
    (partly) derived from ICP-MS/NAA measurements under the assumption of secular
    equilibrium in the U/Th decay chains; values in light blue have to be multiplied by
    2/3. The latter is an approximation for the shortening of the cables before deployment
    in \gerda.
  }\label{tab:bkg:ph2p:assay}
  \addfontfeatures{Numbers=Tabular}
\definecolor{TolGray}      {HTML}{BBBBBB}
\definecolor{TolPaleGray}  {HTML}{DDDDDD}

\definecolor{TolBriBlue}   {HTML}{4477AA}
\definecolor{TolBriCyan}   {HTML}{66CCEE}
\definecolor{TolBriGreen}  {HTML}{228833}
\definecolor{TolBriYellow} {HTML}{CCBB44}
\definecolor{TolBriRed}    {HTML}{EE6677}
\definecolor{TolBriPurple} {HTML}{AA3377}

\definecolor{TolHCYellow}  {HTML}{DDAA33}
\definecolor{TolHCRed}     {HTML}{BB5566}
\definecolor{TolHCBlue}    {HTML}{004488}

\definecolor{TolVibBlue}   {HTML}{0077BB}
\definecolor{TolVibCyan}   {HTML}{33BBEE}
\definecolor{TolVibTeal}   {HTML}{009988}
\definecolor{TolVibOrange} {HTML}{EE7733}
\definecolor{TolVibRed}    {HTML}{CC3311}
\definecolor{TolVibMagenta}{HTML}{EE3377}

\definecolor{TolMutIndigo} {HTML}{332288}
\definecolor{TolMutCyan}   {HTML}{88CCEE}
\definecolor{TolMutTeal}   {HTML}{44AA99}
\definecolor{TolMutGreen}  {HTML}{117733}
\definecolor{TolMutOlive}  {HTML}{999933}
\definecolor{TolMutSand}   {HTML}{DDCC77}
\definecolor{TolMutRose}   {HTML}{CC6677}
\definecolor{TolMutWine}   {HTML}{882255}
\definecolor{TolMutPurple} {HTML}{AA4499}

\definecolor{TolLigBlue}   {HTML}{77AADD}
\definecolor{TolLigCyan}   {HTML}{99DDFF}
\definecolor{TolLigMint}   {HTML}{44BB99}
\definecolor{TolLigPear}   {HTML}{BBCC33}
\definecolor{TolLigOlive}  {HTML}{AAAA00}
\definecolor{TolLigYellow} {HTML}{EEDD88}
\definecolor{TolLigOrange} {HTML}{EE8866}
\definecolor{TolLigPink}   {HTML}{FFAABB}

\newcommand{\rcpg}{\rowcolor{TolPaleGray}}
\newcommand{\rclb}{\rowcolor{TolLigBlue}}
\newcommand{\cclm}{\cellcolor{TolLigMint}}
\newcommand{\ccpg}{\cellcolor{TolPaleGray}}
\newcommand{\ccw}{\cellcolor{white}}

\begin{tabular}{rllcccccc}
  Part                  & Material                  & Method        & {$^{228}$Ra (\mubq)} & {\Ra\ (\mubq)} & {\Th\ (\mubq)} & {\Co\ (\mubq)} & {\kvn\ (mBq)} & {\Uh\ (mBq)}         \\
  \midrule
  \rclb\ccw cables      & \ccw Tecnomec 3 mil       & \ccw\g\ spec. & $<153$               & $<203$         & $<126$         & $<21$          & $3.01\pm0.82$ & $<18.0$              \\
  \rcpg\ccw LAr veto    & \ccw outer BCF-91A fibers & \ccw ICP-MS   & $177\pm53$           & $93\pm28$      & $177\pm53$     & --             & $3.1\pm0.9$   & $0.093\pm0.028$      \\
  \rcpg\ccw             & \ccw inner BCF-91A fibers & \ccw ICP-MS   & $24\pm7$             & $12\pm4$       & $24\pm7$       & --             & $0.41\pm0.01$ & $0.012\pm0.004$      \\
  mini-shroud           & coated + glued            &      ICP-MS   & \cclm$19\pm5$        & \cclm$46\pm14$ & \ccpg$19\pm5$  & --             & $>1.8$        & \ccpg$0.046\pm0.014$ \\
  \bottomrule
\end{tabular}

\end{sidewaystable}

% vim: tw=90
