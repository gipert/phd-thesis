%! TEX root = ../main.tex

\chapter{Monte Carlo simulations and probability density functions}%
\label{apdx:magepdfs}

Background components that were identified in the energy spectra or in
radio-purity screening measurements \cref{assay} are simulated
using the \mage\ software~\cite{Boswell2011} based on
\geant~\cite{Agostinelli2002, Allison2006, Allison2016}.  The \gerda\
\phasetwo\ detectors, their arrangement in seven strings as well as the
LAr instrumentation are implemented into \mage. A graphic rendering of
the relevant implemented hardware components is presented
in~\cref{fig:setup:magevolumes}
\newpar
Simulations of radioactive contaminations
in the following hardware components are performed: in the bulk and on
the \pplus\ and \nplus\ surfaces of the germanium detectors, in the LAr,
detector holder bars and plates, nylon mini-shrouds, LAr veto system
(i.e.~the fiber shroud, SiPMs, copper shrouds and photomultipliers) and
in the signal and high-voltage flexible flat cables. The primary
spectrum of the two electrons emitted in the \nnbb\ decay is sampled
according to the distribution given in reference~\cite{Tretyak1995}
implemented in \decayzero~\cite{Ponkratenko2000}. Note that the
thickness of the detector assembly components are significantly smaller
than the mean free path of the relevant simulated \g-particles in
the given material, thus, no significant difference can be expected
between the resulting spectra of bulk and surface contaminations. The
detectors \nplus\ contact thicknesses are implemented according to the
values reported in references~\cite{Agostini2013a, Agostini2019}.
\newpar
The \kvz\ decays (except for surface contaminations) are simulated
homogeneously distributed in the relevant LAr volume. The following LAr
volumes are chosen for the background model: the first is a cylinder
centered on the detector array ($h=250$~cm, $r=100$~cm, simply referred
to as ``homogeneous'' or abbreviated to ``hom.''~in the following)
subsequently divided into the volume enclosed by the mini-shrouds and
the remaining one (outside the mini-shrouds); the second is a cylinder
($h=100$~cm, $r=25$~cm) positioned just above the array and the
remaining seven are smaller cylinders ($h=20$~cm, $r=5$~cm), each one
positioned just above each of the seven detector strings.
\newpar
On top of the \mage\ simulations a post-processing step is performed to
compute the probability density functions (pdfs) used to model the
\gerda\ data in the statistical analysis. This includes folding in
run-time dependent information, i.e.~the detector status in each physics
run, the finite energy resolution and threshold of each detector. All
pdfs presented in the following are computed using the run-time
parameters of the data sets. A selection of the pdfs projected in energy
space and normalized to the number of simulated primary decays, are
displayed in \cref{fig:bkg:raw:ph2:pdfs:gmodel}
and \cref{fig:bkg:raw:ph2:pdfs:gmodel2}.

For the potassium tracking analysis pdfs binned in detector space are
used to model the data. The rotationally symmetric single-detector pdfs
for the \kvn\ and \kvz\ energy windows are shown in
\cref{fig:bkg:raw:ph2:pdfs:kmodel:K42} and
\cref{fig:bkg:raw:ph2:pdfs:kmodel:K40}.  For two-detector events the
same representation style as in~\cref{fig:kmodel:spc} is used:
projections of the two-dimensional histograms on their axis are summed,
such that each two-detector event enters the final histogram twice, in
the two bins associated to the respective detectors. They can be found
in \cref{fig:bkg:raw:ph2:pdfs:kmodel} together with the single-detector
pdfs of the rotationally asymmetric components. 

Common features can be noticed across the multitude of histogram shapes.
The event rate in single-detector data is generally higher in coaxial
detectors, due to their larger mass compared to BEGe detectors ---
maximal correlation between event rate and detector-by-detector exposure
can be found in the \nnbb\ pdf in~\cref{fig:bkg:raw:ph2:pdfs:kmodel:K42}. This
feature is generally lost in the two-detector data: the coaxial
detectors larger volume allows to stop more efficiently
$\gamma$-particles that would otherwise escape and eventually deposit
energy in a second detector. Other similarities between different pdfs
can be attributed to detectors live-times, like in the case of
\texttt{GD91C}, which was inactive for a large fraction of the
\phasetwo\ exposure and thus generally registers a low number of counts.
The effects of asymmetrically distributed background contaminations are
easily recognizable in the shape of the pdfs.  Impurities located above
the detector array are mostly seen by the upper most detectors in each
string as can be seen for \kvn\ in the front-end electronics
in~\cref{fig:bkg:raw:ph2:pdfs:kmodel:K40} and
in~\cref{fig:bkg:raw:ph2:pdfs:kmodel:M2K40} and for \kvz\ above each
mini-shroud (see~\cref{fig:bkg:raw:ph2:pdfs:kmodel:K42sep} and
\cref{fig:bkg:raw:ph2:pdfs:kmodel:M2K40sep}). Rotationally asymmetric
components are mostly evident in a single string, see for example \kvn\
in single mini-shrouds in \cref{fig:bkg:raw:ph2:pdfs:kmodel:K40sep} and
\cref{fig:bkg:raw:ph2:pdfs:kmodel:M2K40sep}.

\begin{figure}
  \centering
  \subfloat[%
    \kvn\ in different setup locations and \nnbb\ in Ge,
    \Mokvn\ data set.\label{fig:bkg:raw:ph2:pdfs:kmodel:K40}%
  ]{\includegraphics[width=0.48\textwidth]{plots/bkg/raw/ph2/pdfs/kmodel-pdfs-K40.pdf}}
  \hfill
  \subfloat[%
    \kvn\ located close to each single mini-shroud, \Mokvn\
    data set.\label{fig:bkg:raw:ph2:pdfs:kmodel:K40sep}%
  ]{\includegraphics[width=0.48\textwidth]{plots/bkg/raw/ph2/pdfs/kmodel-pdfs-K40-sep.pdf}}

  \subfloat[%
    \kvn\ in different setup locations, \Mtkvn\
    data set.\label{fig:bkg:raw:ph2:pdfs:kmodel:M2K40}%
  ]{\includegraphics[width=0.48\textwidth]{plots/bkg/raw/ph2/pdfs/kmodel-pdfs-K40-M2.pdf}}
  \hfill
  \subfloat[%
    \kvn\ located close to each single mini-shroud, \Mtkvn\
    data set.\label{fig:bkg:raw:ph2:pdfs:kmodel:M2K40sep}%
  ]{\includegraphics[width=0.48\textwidth]{plots/bkg/raw/ph2/pdfs/kmodel-pdfs-K40-sep-M2.pdf}}

  \caption{%
    pdfs binned in detector space for the \kvn\ tracking analysis. 
    All pdfs are normalized to the number of simulated primary decays.
  }\label{fig:bkg:raw:ph2:pdfs:kmodel:K40}
\end{figure}

\begin{figure}
  \subfloat[%
    \kvz\ in different setup locations and \nnbb\ in Ge,
    \Mokvz\ data set.\label{fig:bkg:raw:ph2:pdfs:kmodel:K42}%
  ]{\includegraphics[width=0.48\textwidth]{plots/bkg/raw/ph2/pdfs/kmodel-pdfs-K42.pdf}}
  \hfill
  \subfloat[%
    \kvz\ in LAr above each single mini-shroud, \Mokvz\
    data set.\label{fig:bkg:raw:ph2:pdfs:kmodel:K42sep}%
  ]{\includegraphics[width=0.48\textwidth]{plots/bkg/raw/ph2/pdfs/kmodel-pdfs-K42-sep.pdf}}

  \subfloat[%
    \kvz\ in different setup locations, \Mtkvz\
    data set.\label{fig:bkg:raw:ph2:pdfs:kmodel:M2K42}%
  ]{\includegraphics[width=0.48\textwidth]{plots/bkg/raw/ph2/pdfs/kmodel-pdfs-K42-M2.pdf}}
  \hfill
  \subfloat[%
    \kvz\ in different setup locations, \Mtkvz\
    data set.\label{fig:bkg:raw:ph2:pdfs:kmodel:M2K42sep}%
  ]{\includegraphics[width=0.48\textwidth]{plots/bkg/raw/ph2/pdfs/kmodel-pdfs-K42-sep-M2.pdf}}

  \caption{%
    pdfs binned in detector space for the \kvz\ tracking analysis. 
    All pdfs are normalized to the number of simulated primary decays.
  }\label{fig:bkg:raw:ph2:pdfs:kmodel:K42}
\end{figure}

All \a\ decays in the \Ra\ to \Pbl\ sub-chain and from \Po\ are
simulated on the \pplus\ detector surface separately and for different
thicknesses of the \pplus\ electrode. The \Ra\ chain is simulated
together under the assumption that in each \a\ decay half of the
contamination is lost due to the recoil of the nucleus into the LAr. The
resulting pdfs are displayed in~\cref{fig:bkg:raw:ph2:pdfs:amodel:Po}
and \cref{fig:bkg:raw:ph2:pdfs:amodel:Ra}. The spectra exhibit a peak
like structure with a pronounced low-energy tail.  The maximum is
shifted with respect to the full emission energy due to the thickness of
the \pplus\ contact.  The low-energy tail is characteristic for \a\
decays; the \a\ particle is susceptible to the change in the contact
thickness when penetrating the detector surface under an incident angle
and loses part of its energy before reaching the active detector volume.

% \begin{table}[b]
% \centering
% \caption{%
% Mass of materials which are deployed in the \gerda\ setup and their
% respective mass in the MaGe MC implementation. Fiber shroud and SiPM
% ring deployed masses include the support material and hence show a
% larger discrepancy to the implemented mass. ($^\dagger$ \tetratex
% -coated)} \label{tab:masses}
% \begin{tabular}{lrr}
% \toprule
% \multirow{2}{*}{Part}         & deployed       & MC       \\
%                               & mass [g]       & mass [g] \\
% \midrule
% detector holders              & 640            & 659      \\
% flat cables (80~pc)           & 75.36          & 31       \\
% front-end electronics (11~pc) & 190.5          & 51.96    \\
% mini-shrouds                  & 196.7          & 175      \\
% fiber shroud                  & 4298.4         & 1361     \\
% SiPM ring                     & 74.4           & 3.4      \\
% copper shrouds$^\dagger$      & 1686           & 1584     \\
% \bottomrule
% \end{tabular}
% \label{tab:masses}
% \end{table}

% vim: tw=72 tabstop=2 expandtab shiftwidth=2
