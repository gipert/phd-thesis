%!TEX root = ../main.tex

La ricerca del decadimento doppio beta senza neutrini viene spesso indicata come l'unica
maniera pratica per stabilire la natura della massa del neutrino, una delle particelle
elementari più sfuggenti e intriganti del Modello Standard. La rilevazione di questo raro
decadimento nucleare attribuirebbe al neutrino delle caratteristiche descritte per la
prima volta da Ettore Majorana all'inizio del secolo scorso, mettendo in chiara luce
l'inadeguatezza delle attuali teorie sulla fisica fondamentale. Secondo alcune teorie,
potrebbe anche contribuire a risolvere il mistero dell'asimmetria tra materia e
anti-materia nel nostro universo. Da oltre cinquant'anni, il decadimento doppio-beta senza
neutrini viene ricercato, senza successo, nell'isotopo del germanio formato da 76
nucleoni. L'esperimento \gerda, conclusosi a novembre del \yr{2019}, è stato un pioniere
del campo, avendo dimostrato la maturità di questa tecnica sperimentale per la
realizzazione di un esperimento su larga scala, in grado di arrivare a mettere limiti
dell'ordine di $10^{27}$~anni sulla vita media del decadimento.  Questo lavoro di tesi
vuole mettere in luce gli eccellenti risultati raggiunti in termini di livello di eventi
di fondo, elaborando un modello statistico di questi ultimi. In particolare, verrà
presentata per la prima volta una simulazione Monte Carlo del taglio dell'argon liquido,
basato sulla luce di scintillazione in esso prodotta al passaggio di radiazione
ionizzante. I risultati di questo modello verranno impiegati nello studio della
distribuzione in energia del decadimento doppio beta a due neutrini, allo scopo di
misurarne la vita media e ricercare segnali di nuova fisica provenienti da una eventuale
emissione di Majoroni o dalla presenza di violazioni delle simmetrie di Lorentz e CPT. La
stima della vita media del decadimento doppio beta a due neutrini estratta è di
$(\fillme{?} \pm \stat{\fillme{?}} \pm \syst{\fillme{?}}) \cdot 10^{21}$~anni, mentre il
limite inferiore sulla vita media del decadimento doppio-beta senza neutrini ma con
emissione di un Majorone (indice spettrale $n=1$) ottenuto al 90\% C.L.~è di $\fillme{?}
\cdot 10^{23}$~anni. I risultati superano notevolmente in precisione stime precedenti con
\gesix.

% vim: spelllang=it, tw=90
