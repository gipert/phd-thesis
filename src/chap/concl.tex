%!TEX root = ../main.tex

\chapter*{Conclusions and outlook}%
\label{chap:concl}
\addcontentsline{toc}{chapter}{Conclusions and outlook}

In the course of this thesis work, a path has been drawn from the origins of the \gerda\
background model to a full study of the spectral shape of two-neutrino double-beta decay
events. In \cref{chap:bkg:raw:ph2}, the background model of \gerdatwo\ data before
analysis cuts has been presented to the reader, to demonstrate the accuracy of the Monte
Carlo predictions in reproducing the experimental data. The distribution of the background
in the region of interest for neutrinoless double-beta decay is shown to be uniform,
excluding two \g\ peaks, which are far enough from the Q-value. In
\cref{chap:bkg:lar:ph2}, it has been shown how propagation of optical photons is
implemented into the Monte Carlo framework in order to simulate the liquid argon veto
system. Thanks to this capability, the LAr veto flag has been computed for simulated
background events, and a background model of events after the LAr veto cut has been
developed. This last achievement allowed for a precision analysis of the two-neutrino
double-beta decay energy spectrum, both to measure the process half-life and to search for
new-physics phenomena. No evidence fir

% vim: tw=90
