%!TEX root = ../main.tex

\chapter*{Conclusions and outlook}%
\label{chap:concl}
\addcontentsline{toc}{chapter}{Conclusions and outlook}

In the course of this thesis work, a path has been drawn from the origins of the \gerda\
background model to a full study of the spectral shape of two-neutrino double-beta decay
events. In \cref{chap:bkg:raw:ph2}, the background model of \gerdatwo\ data before
analysis cuts has been presented to the reader, to demonstrate the accuracy of the Monte
Carlo predictions in reproducing the experimental data. The distribution of the background
in the region of interest for neutrinoless double-beta decay is shown to be uniform,
excluding two \g\ peaks, which are far enough from the Q-value. In
\cref{chap:bkg:lar:ph2}, it has been shown how propagation of optical photons is
implemented into the Monte Carlo framework in order to simulate the liquid argon veto
system. Thanks to this capability, the LAr veto flag has been computed for simulated
background events, and a background model of events after the LAr veto cut has been
developed. This last achievement allowed for a precision analysis of the two-neutrino
double-beta decay energy spectrum, both to measure the process half-life and to search for
new-physics phenomena. The experimental estimate of the two-neutrino double-beta decay
half-life $\thalftwoM = (\fillme{?} \pm \fillme{?}) \cdot 10^{21}$~yr is one of the most
precise measurements ever reported in the double-beta decay research field. Limits on
Majoron-emitting neutrinoless double-beta decay modes are set in the
$10^{22}$--$10^{23}$~yr range, competitive with recent experimental estimates by
\kamlandzen\ and EXO-200. The most stringent half-life lower limit of $\fillme{?} \cdot
10^{23}$~yr at 90\% C.L.~is set on the Majoron-emitting double-beta decay modes with
spectral index $n=1$. The latter converts to an upper limit on the coupling constant \ga\
of \fillme{?}--\fillme{?}~$10^{-5}$~GeV. A search for violations of the Lorentz symmetry
and CPT theorem effects is also performed on the two-neutrino double-beta decay spectrum.
No significant evidence of the spectral deformation is found, and a two-sided 90\%
C.L.~interval of $\fillme{?}~\text{GeV} < \aofM < \fillme{?}~\text{GeV}$ is placed on the
relevant coefficient within the Standard Model Extension.
\newpar
In conclusion of this research work, I would like to spend few words on learned lessons
and possible ideas on how to improve the quality of the science program in future
experimental efforts~---~the LEGEND program, in particular. Building a Monte Carlo model
of the liquid argon veto has proven to be a challenging task. From one side, optical
specifications of materials commonly used in low-background cryogenic experiments are
poorly known in the vacuum ultra-violet regime. A particularly serious problem is posed by
the lack of a precise measurement of the germanium reflectivity at the LAr scintillation
light typical wavelengths. Dedicated measurements campaigns commissioned by the LEGEND
collaboration are ongoing.  Properties of liquid argon, e.g.~the attenuation length,
should be also determined with dedicated devices and monitored during data taking. Ideas
on how to put this in practice are being developed for the LEGEND experiment. From the
other side, optical simulations are exceptionally demanding from the computational point
of view, on average CPUs. The issue has been partly overcome by generating a
light-detection probability map, that effectively encapsulates the LAr veto model and can
be applied to existing background simulations obtained without photon tracking.
Nevertheless, generating a single map requires tens of thousands of CPU hours, and its
characterization is therefore problematic. Experiments in which massive simulations of
optical processes are imperative usually make use of Graphical Processing Units
(GPUs)~\cite{Merck2012, Blyth2019} --- a possibility that could be considered by the
LEGEND collaboration. A more efficient way to simulate light propagation in liquid argon
would speed up the development of a reliable background model after the LAr veto cut and
the subsequent analysis of the \nnbb\ distribution.
\newpar
The second issue regards the active volume of germanium detectors. A precise knowledge of
this detector parameter is mandatory to extract an unbiased measure of the \nnbb-decay
half-life. Evidence for unreliable active volume fractions of \bege\ detectors has emerged
from the study of the \nnbb\ distribution in different detector types and a preliminary
analysis of \Arl\ data.  The detectors have been indeed precisely characterized after
being fabricated, but were then stored at room temperature for a long period before
deployment in liquid argon. At room temperature, the detector dead layer grows at a rate
which has never been rigorously determined. The \Arl\ data set offers the unique
opportunity to calibrate the detector active volume at the same experimental conditions in
which the physics data is recorded. A careful analysis of the \Arl\ energy spectrum is
strongly advised to correct and reduce the uncertainty of the \nnbb\ half-life estimate
presented in this work and to measure the dead-layer growth rate.
\newpar
The ultimate achievement for a \gesix\ neutrinoless double-beta decay experiment would be
to build a model of the background after the pulse-shape discrimination cut. The results
would be extremely relevant for the neutrinoless signal search, which is performed after
this cut, and the two-neutrino double-beta decay distribution analysis. As a matter of
fact, the \gerda\ energy spectrum after PSD consists of a nearly-pure double-beta decay
event sample. Unfortunately, computing the PSD flag for Monte Carlo events is currently a
challenging task, as pulse-shape simulations are still under heavy development. Heuristic
PSD cuts based on the width of the distribution of interaction centers in germanium
detectors show only a qualitative agreement. An intense pulse-shape simulation program,
started in the \gerda\ and \majorana\ collaborations, is ongoing for the LEGEND project.

\chapendgliph{}

% vim: tw=90
