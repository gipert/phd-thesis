%!TEX root = ../main.tex

\chapter{\texorpdfstring{Time distribution of \a\ events}{Time distribution of alpha-events}}%
\label{apdx:timealpha}

The time distribution of \Po\ decays is well known to be exponential, however, in the
presence of a \Pbl\ contamination a constant contribution can also be observed. \Pbl,
decaying to \Po, feeds a constant \Po\ component once their decay rates stabilize in a
secular equilibrium. To disentangle the two we fit the time distribution of events with
energies between 3.5~MeV and 5.25~MeV with a constant $C$ and an exponential function:
\[
  f(t) = C + N \exp\left( - \frac{\log2}{T_{1/2}}t \right)
\]
where $T_{1/2}=(138.4\pm0.2)$~days is the half-life of \Po. We use a Poisson likelihood
function corrected for data acquisition dead time~\cite{Cleveland1983} and model the time
bin content as follows
\[
  \nu_i = f_i^{\mathrm{LT}}
  \left\{ C \delta t + N \tau
    \left[
      \exp\left( -\frac{t_0 + i \delta t}{\tau} \right)  -
      \exp\left( -\frac{t_0 + (i+1) \delta t}{\tau} \right)
    \right]
  \right\}
\]
$C$ and $N$ are the amplitudes of the constant and the exponentially decaying components
and are the only two free fit parameters.  $f_i^{\mathrm{LT}}$ is the live-time fraction
in time-bin $i$ which is estimated from injected test pulser events, $\delta t$ is the
time-bin width and $\tau = T_{1/2} / \log2$.

The log-likelihood can be written as a sum:
\[
  \log \mathcal{L}_\alpha^\text{time}(C,N \,|\, n) =
  \sum_{i=1}^{N_\text{bins}} n_i \cdot
  \log\nu_i - \nu_i - \log n_i!
\]
We select only detectors that were ON or in anti-coincidence mode\footnote{Detectors in
anti-coincidence are not well energy-calibrated and generally discarded in data analysis.
Here, we are not interested in the precise energy of an event because the selected energy
window is large with respect to a possible miscalibration.} in the full data taking
period. In this way we avoid bias due to selection or deselection of particularly
contaminated detectors. Furthermore, we exclude the initial data-taking period between
December 2015 to January 2016 from the following analysis because of detector
instabilities after the \phasetwo\ upgrade works. The analyzed data span from the end of
January 2016 to the end of November 2019 and are split into five data sets according to
detector type (\bege, \scoax\ and \icoax) and experimental phase (\phaseone\ and
\phasetwop). The fit results are shown in \autoref{fig:bkg:raw:timealpha:results}. For the
\bege\ data set we find that about half of the initial contamination decays exponentially
while for the \scoax\ data set the ratio of $N$ to $C$ is about 5 to 1. After several \Po\
half-lives we expect a stable rate of $\sim1~\alpha$/day in either data set.
\fillme{comment}

\begin{figure}
  \centering
  \includegraphics{plots/bkg/raw/ph2/results/amodel/amodel-timealpha.pdf}
  \caption{%
    \a\ events time distribution in the $[3500,5250]$~keV energy range during the whole
    \phasetwo\ data taking. Each data point correspond to events cumulated in 20~days of
    effective run time. Five data sets are considered: 27 \bege\ detectors in \phasetwo\ and
    \phasetwop\ (top panel), 7 \scoax\ detectors in \phasetwo\ and \phasetwop, and 4
    \icoax\ detectors in \phasetwop\ (bottom panel). An exponential model is fit to each
    data set and the results are overlaid on the plot.
  }\label{fig:bkg:raw:timealpha:results}
\end{figure}

% vim: tw=90
