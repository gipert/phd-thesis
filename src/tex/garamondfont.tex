%!TEX root = ../main.tex

% GARAMOND FONT TEMPLATE. COMMANDS:
% \liningnums, \oldstylenums, \tabnums      : different styles for numbers
% \initial{<letter>}{<text in sc>}          : create the first fancy big letter: A D F G L N O Q I
% \decorationA \decorationB \decorationC    : decorations
% \textit \textsw \textsc \textinit         : available shapes
% \nicefrac                                 : nice inline fraction style
% \darkred                                  : nice dark red colour
%
%
% Here you can set global settings
\usepackage{fontspec}
\defaultfontfeatures{%
  RawFeature={%
    +calt   % *Contextual alternates
    ,+clig  % *contextual ligatures
    ,+ccmp  % *composition & decomposition
    %,+lnum  % lining numbers
    %,+tnum  % tabular numbers
    %,+frac  % nice fractions
    ,+tlig  % 'tex-ligatures': `` '' -- --- !` ?` << >>
    %,+cv06  % narrow guillemets
    % ...
  }%
}
\setmainfont{EBGaramond12-Regular}[
  ItalicFont = EBGaramond12-Italic
]
% use typewriter computer modern as monospaced font
%\setmonofont{cmuntt.otf}[Scale=MatchLowercase]
\renewcommand\ttdefault{cmtt}
\newcommand{\textsw}[1]{\textit{\addfontfeature{RawFeature=+swsh}#1}}
\newcommand{\tabnums}[1]{{\addfontfeature{RawFeature=+tnum}#1}}
\newcommand{\decorationA}{{\Huge\char"E001\char"E002}}
\newcommand{\decorationB}{\Huge\char"2619\char"2767}
\newcommand{\decorationC}{\Huge\char"2766}
\newcommand{\nicefrac}[1]{{\addfontfeature{RawFeature=+frac}#1}}
\newcommand{\darkred}[1]{{\addfontfeature{Color=980000}#1}}
% initials
\usepackage{lettrine}
\newfontface{\initsh}{EBGaramond-Initials}
\newcommand{\textinit}[1]{{\initsh #1}}
\newcommand{\initial}[2]{\lettrine[lines=3, depth=1, findent=2pt, nindent=0em]{\initsh #1}{#2}}
% math
\usepackage[cmbraces, varg, libertine]{newtxmath}
\usepackage{bm}\renewcommand{\mathbf}{\bm}
\renewcommand{\emph}[1]{\textsw{#1}}
