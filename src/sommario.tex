%!TEX root = main.tex

Nel quadro dello studio dei cosiddetti fenomeni di nuova fisica oltre la ben consolidate
teorie di fisica delle particelle e delle alte energie, la violazione del numero leptonico
viene spesso indicata come una delle possibilit\'a pi\'u feconde, dal punto di vista
teorico. Se verificata sperimentalmente, l'ipotesi che questa tipologia di particelle
elementari, i leptoni, non sia coinvolta in maniera omogenea nei processi fondamentali della
natura confermerebbe l'inadeguatezza delle attuali teorie e potrebbe contribuire a
risolvere il mistero dell'asimmetria tra materia e anti-materia nel nostro universo. Uno
dei processi elementari tramite il quale i fisici sono alla ricerca di questo fenomeno e' il
decadimento doppio-beta, osservabile in particolari nuclei atomici. Questo speciale
processo coinvolge direttamente il neutrino, uno dei leptoni pi\'u sfuggenti e intriganti
nelle sue manifestazioni fisiche.

% vim: spelllang=it, tw=90
